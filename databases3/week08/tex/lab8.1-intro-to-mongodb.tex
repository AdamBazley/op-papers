\documentclass{article}
\usepackage{graphicx}
\usepackage{wrapfig}
\usepackage{inconsolata}
\usepackage{enumerate}
\usepackage{verbatim}
\usepackage{hyperref}
\usepackage[parfill]{parskip}
\usepackage[margin = 2.5cm]{geometry}

\usepackage[T1]{fontenc}


\begin{document}

\title{Lab 8.1: Introduction to MongoDB\\ IN705 Databases Three}
\date{}
\maketitle

\section*{Introduction}
Relational databases are, and are likely to remain, the preferred way to handle object persistence in applications.  Relational databases are especially good at providing transactional integrity and at processing ad-hoc queries.  

For some applications, however, other databases may work well, particularly when there is a need for easy scalability or to avoid fixed schema.  A variety of \emph{NoSQL} databases have emerged recently that address these needs.  One such database is MongoDB\footnote{See \url{http://www.mongodb.org}}.

MongoDB is a \emph{document database} that stores its data in \emph{BSON}, or Binary JSON, format.  JavaScript, rather than SQL, is used to interact with the database.

MongoDB is used in a number of notable large deployments for organisations like SAP, Codecademy, and CERN.

\section{Install MongoDB}
Although MongoDB has its own package repository, the standard Ubuntu package suits our needs.  Install it on your EC2 server with the command

\texttt{sudo apt-get install mongodb}

\section{Using the Mongo shell}
Start MongoDB with the command

\texttt{mongo lab8}

MongoDB stores data in \emph{collections} rather than tables.  Unlike tables,
collections have no fixed schemata and can be created by simply inserting a record into one, like this:

\begin{verbatim}

db.cities.insert( {
  name: "Dunedin",
  population: 126000,
  mayor: "Dave Cull",
  famous_for: ["scarfies", "shit weather"]
})

\end{verbatim}

Now you can see that the \texttt{cities} collection exists by entering the \texttt{show collections} command.  You can view the items in the collection by using \texttt{db.cities.find()}.

The language of the MongoDB shell is Javascript.  You can see the source of any function by typing its name without brackets, like this:

\texttt{db.cities.insert}

You can also write your own functions in JavaScript, like this:

\begin{verbatim}
function addCity(name, population, mayor, famous_for) {
    db.cities.insert({ name: name, population: population, 
                       mayor: mayor, famous_for: famous_for});
}
\end{verbatim}

Now we can add cities like this:

\texttt{addCity("Christchurch", 366000, "Lianne Dalziel", ["earthquakes"]) }

Since MongoDB collections don't have to adhere to schemata, you can insert a
city like this:

\texttt{db.cities.insert(\{name: "Melbourne", country: "Australia"\})}

\section{Querying}
MongoDB provides decent support for ad hoc queries.  For example, to find 
cities with populations under 200,000, we could enter

\texttt{db.cities.find(\{population: \{\$lt: 200000\}\})}

We can find Dunedin by entering

\texttt{db.cities.find(\{name: "Dunedin"\})}

\section{Updating}
Updating is a bit tricky.  For example


\texttt{db.cities.update(\{name: "Dunedin"\}, \{population: 125000\})}

probably doesn't do what you want.  Instead, use the \texttt{\$set} operator like this:


\texttt{db.cities.update(\{name: "Dunedin"\}, \{\$set: \{population: 125000\}\})}

\section{Deleting}
You can remove records from a collection by passing in arguments that will find the correct records, like this:

\texttt{db.cities.remove(\{name: "Christchurch"\})}

\section{Homework}
Explore more of the capabilities of MongoDB by consulting the relevant chapter in the \emph{Seven Databases} book or online documents.

Since Mongo data records are scored in a JSON-like format, it should be easy to see how our splatter data could be transferred to it.  Sketch out a plan to do this.


\end{document}
