\documentclass{article}
\usepackage{enumerate}
\usepackage{verbatim}
\usepackage{hyperref}
\usepackage[parfill]{parskip}
\usepackage[margin = 2.5cm]{geometry}

\usepackage[T1]{fontenc}


\begin{document}

\title{Lab 14.2: Database Design and Performance\\ IN705 Databases Three}
\date{}
\maketitle

\section*{Introduction}
When we are seeking to optimise the performance of our database applications
we know that the most productive place to look for performance gains is in the database 
itself.  One way to improve performance is to pay attention to these issues in database design.

\section{Table size}
In modern software development we don't typically worry about the volume of data
we store since storage space is plentiful on typical machines.  However, there is
some performance value in keeping the size of our database tables small.  This is 
becasue a smaller table that may be held entirely in our server's RAM can be 
accessed more quickly than one that must be retrieved from disk.  For tables
that can't be stored in RAM, it is desirable that their indexes (see below) can be.

We can only do so much to control the size of our data, since we don't control the
real-world information that we model.  Two ways to do this are:

\begin{itemize}
	\item Choose appropriate column types
	\item Design tables carefully
\end{itemize}

As an example of the second, suppose that we store data about system users.  Some
information is accessed frequently, e.g. name, password, email address.  Other
items may be accessed only infrequently.  It may be worthwhile to split the data into two
smaller tables in this case.

\section{Column types}
Choosing column types wisely can help application performance.

\subsection{Character types}
This one is tricky.  PostgreSQL, like most database systems, has three
character types.

\begin{itemize}
	\item character, for fixed-width strings
	\item character varying, for variable length strings
	\item text, for large variable length strings
\end{itemize}


\end{document}
