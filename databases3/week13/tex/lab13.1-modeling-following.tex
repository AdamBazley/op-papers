\documentclass{article}
\usepackage{enumerate}
\usepackage{verbatim}
\usepackage{hyperref}
\usepackage[parfill]{parskip}
\usepackage[margin = 2.5cm]{geometry}

\usepackage[T1]{fontenc}


\begin{document}

\title{Lab 13.1: Modeling Following/Followers\\ IN705 Databases Three}
\date{}
\maketitle

\section*{Introduction}
Now that we have our Riak data store in place and our models started, today we will work on adding the following/followers relationship.  Each user mainatains a list of users followed and a list of users who follow it.  Adding a following relationship between to users, where \emph{User A} follows \emph{User B} consists of adding \emph{User B} to \emph{User A's} \texttt{follows} list and adding \emph{User A} to \emph{User B's} \texttt{followers} list.  Deleting the relationship is a matter of deleting the respective users from each others' lists.

One additional consideration is that in some cases we are creating a new list (when a user has not yet followed or been followed by anyone) and in other cases we are appending to an exisiting list.

\section{Implementing the follows relationship}
We will add this functionality from the outside in by first writing our controller's \texttt{add\_follows} method.  The basic plan is to

\begin{enumerate}
	\item Get the following and followed user;
	\item Append the user id's to the appropriate lists (creating the list if it does not exisit);
	\item Update both users.
\end{enumerate}

The \texttt{add\_follows} method looks like this:

\emph{file:  app/controllers/users\_controller.rb}

\begin{verbatim}
 # POST /users/follows
 def add_follows
   db = UserRepository.new(Riak::Client.new)
   @follower = db.find(params[:id])
   @followed = db.find(params[:follows_id])

   if db.follow(@follower, @followed)
     head :no_content
   else
     render json: "error saving follow relationship" , status: :unprocessable_entity
   end
 end

\end{verbatim}

We can see that we need to add the \texttt{follow} method to the \texttt{UserRepository} class.

\emph{file:  app/models/user\_repository.rb}

\begin{verbatim}
def follow(follower, followed)
  if follower.follows
    follower.follows << followed.email
  else
    follower.followed = [followed.email]
  end

  if followed.followers
    followed.followers << follower.email
  else
    followed.followers = [follows.email]
  end
  
  update(followed)
  update(follwer)
end

\end{verbatim}

Unfollowing is basically a mirror image of the \texttt{follow} method and is left as an exercise.  The \texttt{show\_follows} method also needs modification to work with Riak but is also straightforward.


\end{document}
