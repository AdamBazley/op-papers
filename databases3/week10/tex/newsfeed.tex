% Beamer slide template prepared by Tom Clark <tom.clark@op.ac.nz>
% Otago Polytechnic
% Dec 2012

\documentclass[10pt]{beamer}
\usetheme{CambridgeUS}
\usepackage{graphicx}
\usepackage{fancyvrb}


\title{Using MapReduce to Retrieve Our News Feed}

\author[IN705]{Databases Three}
\institute[Otago Polytechnic]{
  Otago Polytechnic \\
  Dunedin, New Zealand \\
}
\date{}
\begin{document}

%----------- titlepage ----------------------------------------------%
\begin{frame}[plain]
  \titlepage
\end{frame}



%----------- slide --------------------------------------------------%
\begin{frame}
  \frametitle{Last time}

 \begin{itemize}
  \item We saw how to use MapReduce to perform complex queries.
  \item We supply a \emph{map} function that gets data from documents.
  \item We supply a \emph{reduce} function that aggregates the map results.
  \item We supply additional options as required.
 \end{itemize}

\end{frame}


%----------- slide --------------------------------------------------%
\begin{frame}
  \frametitle{Retrieving our feed}

 Map stage
 \begin{itemize}
	 \item Iterate over the users followed by a given user.
         \item \emph{Emit} the splatts for each user in the list.
 \end{itemize}

\end{frame}


%----------- slide --------------------------------------------------%
\begin{frame}[fragile]
  \frametitle{Map code}

 \begin{verbatim}
 var map = function() {
   if(this.splatts) {
     emit("feed", {"list": this.splatts})
   }
 }
 \end{verbatim}

\end{frame}


%----------- slide --------------------------------------------------%
\begin{frame}
  \frametitle{Retrieving our feed}
 Reduce stage
 \begin{itemize}
  \item Merge our lists of splatts together.
  \item At this stage, we leave them unsorted.
 \end{itemize}

\end{frame}



%----------- slide --------------------------------------------------%
\begin{frame}[fragile]
  \frametitle{Reduce code}

 \begin{verbatim}
 var reduce = function(key, values) {
     var myfeed = {"list": []};
     values.forEach(function(v) {
         myfeed.list = myfeed.list.concat(v.list);
     });
     return myfeed;
 }
	     \end{verbatim}

\end{frame}


%----------- slide --------------------------------------------------%
\begin{frame}
  \frametitle{Retrieving our feed}

  Additional options
 \begin{itemize}
  \item Output
  \item Query
  \item Finalisation
 \end{itemize}


\end{frame}
%----------- slide --------------------------------------------------%
\begin{frame}[fragile]

  \frametitle{Output}

 \begin{verbatim}
  output:  {inline: 1}
 \end{verbatim}

\end{frame}


%----------- slide --------------------------------------------------%
\begin{frame}[fragile]
  \frametitle{Query}

 \begin{verbatim}
 query: {_id:  {$in: db.users.findOne(
   {_id: ObjectId("5416717562696259b8000000")}).follow_ids }
 }
 \end{verbatim}

\end{frame}


%----------- slide --------------------------------------------------%
\begin{frame}[fragile]
  \frametitle{Finalisation}

 \begin{verbatim}
 var finalise = function(key, val) {
     var mylist = val.list;
     if(mylist) {
         mylist.sort(function(a, b) { 
           return b.created_at - a.created_at});
      }
     return {"list": mylist};
 }

 \end{verbatim}

\end{frame}

%----------- slide --------------------------------------------------%
\begin{frame}[fragile]
  \frametitle{Putting it all together}

 \begin{verbatim}
 db.users.mapReduce(map, reduce,
    {
    out: {inline: 1},
    finalize: finalise,
    query: {_id:  {$in: db.users.findOne(
       {_id: ObjectId("5416717562696259b8000000")}).follow_ids }
        }
    })
 \end{verbatim}

\end{frame}


%----------- slide --------------------------------------------------%
\begin{frame}[fragile]
  \frametitle{In Ruby}

  Define strings containing the JavaScript code for map, reduce, and finalise.
  Then, you can do this:
 \begin{verbatim}
  user = User.find(params[:id])
  result = User.in(id: user.follow_ids).map_reduce(map, reduce).
           out(inline: true).finalize(finalise)
  render json: result.entries[0][:value][:list]
 \end{verbatim}

\end{frame}


\end{document}
