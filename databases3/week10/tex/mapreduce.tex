% Beamer slide template prepared by Tom Clark <tom.clark@op.ac.nz>
% Otago Polytechnic
% Dec 2012

\documentclass[10pt]{beamer}
\usetheme{CambridgeUS}
\usepackage{graphicx}
\usepackage{fancyvrb}


\title{MapReduce}

\author[IN705]{Databases Three}
\institute[Otago Polytechnic]{
  Otago Polytechnic \\
  Dunedin, New Zealand \\
}
\date{}
\begin{document}

%----------- titlepage ----------------------------------------------%
\begin{frame}[plain]
  \titlepage
\end{frame}



%----------- slide --------------------------------------------------%
\begin{frame}
  \frametitle{Problem: counting splatts}
 Suppose we want to count the number of splatts in our collection.
 \begin{itemize}
	 \item Relational : \texttt{select count(*) from splatts;}
	 \item Document : If we had a splatts collection, we could
		 do this - \texttt{db.splatts.count()}  - but
		 we don't have one.
	 \item We need to tell MongoDB how to find and count our
		 splatts.
		
 \end{itemize}

\end{frame}



%----------- slide --------------------------------------------------%
\begin{frame}[fragile]
  \frametitle{The data}

 \begin{verbatim}
 {
   "_id" : ObjectId("5416717562696259b8010000"),
   "email" : "imp@casterlyrock.com",
   "follower_ids" : [ ObjectId("5416717562696259b8000000") ],
   "name" : "Tyrion Lannister",
   "splatts" : [
      {
       "_id" : ObjectId("5416727962696259b8030000"),
       "body" : "user 2 splatt 2",
       "created_at" : ISODate("2014-09-15T05:00:41.376Z")
      } 
   ] 
}

 \end{verbatim}

\end{frame}


%----------- slide --------------------------------------------------%
\begin{frame}
  \frametitle{The plan}

 \begin{enumerate}
  \item Iterate over the users\footnote{Our data set is small, but this could be parallelised over
	  a large data set.};
  \item Count the splatts belonging to each one;
  \item Add up the counts.
 \end{enumerate}

 This is a common pattern, and it has a name: MapReduce.
\end{frame}


%----------- slide --------------------------------------------------%
\begin{frame}
  \frametitle{MapReduce}

 \begin{enumerate}
  \item Map:  For each user, count the splatts.
  \item Reduce: Sum up the indvidual counts.
 \end{enumerate}

\end{frame}


%----------- slide --------------------------------------------------%
\begin{frame}[fragile]
  \frametitle{MapReduce implementation}

  We need to supply JavaScript code to MongoDB telling it
  how to perform the Map and Reduce steps
 \begin{itemize}
  \item We write a map function that is called one time on
	  each user in the collection.  It will emit the 
	  count of each user's splatts.
  \item We write a reduce function that sums up the output from the 
	  map function.
  \item We supply both of these functions to MongoDB's \texttt{mapReduce} command.
 \end{itemize}

\end{frame}



%----------- slide --------------------------------------------------%
\begin{frame}[fragile]
  \frametitle{Map}

 \begin{verbatim}
var map = function() { 
    var length = 0;
    if(this.splatts) {
        length = this.splatts.length
    }
    emit ("count", length); 
};
 \end{verbatim}

\end{frame}


%----------- slide --------------------------------------------------%
\begin{frame}[fragile]
  \frametitle{Map results}

 After applying the map to a collection with 3 users, the result set
 may look like this:
 \begin{verbatim}
 results = [
     {key: "count", value: 4 },
     {key: "count"  value: 3 },
     {key: "count", value: 3 }
     ]
 \end{verbatim}

\end{frame}


%----------- slide --------------------------------------------------%
\begin{frame}[fragile]
  \frametitle{Reduce}

 \begin{verbatim}
var reduce =  function(key, val) {
    var data =  0;
    val.forEach(function(v) {
        data += v;
    });
    return data;
}
 \end{verbatim}

This reduce function will be called one time for each key value in the map results.
\end{frame}



%----------- slide --------------------------------------------------%
\begin{frame}[fragile]
  \frametitle{MapReduce}

 We use our \texttt{map} and \texttt{reduce} functions in a call to MongoDB's 
 \texttt{mapReduce} command.
 \begin{verbatim}
  db.users.mapReduce(
    map,
    reduce,
    {
      out: {inline: 1}
    }
  }


 \end{verbatim}

\end{frame}


%----------- slide --------------------------------------------------%
\begin{frame}[fragile]
  \frametitle{The final result}

 \begin{verbatim}
{
  "results" : [
    {
      "_id" : "count",
      "value" : 10
    }
  ],
  "timeMillis" : 0,
  "counts" : {
    "input" : 3,
    "emit" : 3,
    "reduce" : 1,
    "output" : 1
  },
  "ok" : 1,
}

 \end{verbatim}

\end{frame}


%----------- slide --------------------------------------------------%
\begin{frame}[fragile]
  \frametitle{In Ruby}

 The Mongoid library provides access to MongoDB's MapReduce functionality:

 \begin{verbatim}
 map = %Q{ function() { 
      var length = 0;
      if(this.splatts) {
          length = this.splatts.length
      }
      emit ("count", length); 
  } 
}
 reduce =  %Q{ function(key, val) {
    var data =  0;
    val.forEach(function(v) {
        data += v;
    })
    return data;
 }
}
 \end{verbatim}

\end{frame}


%----------- slide --------------------------------------------------%
\begin{frame}[fragile]
  \frametitle{In Ruby}

 \begin{verbatim}

User.map_reduce(map,reduce).out(inline: true)
 \end{verbatim}

\end{frame}


%----------- slide --------------------------------------------------%
\begin{frame}
  \frametitle{Today's lab}

 \begin{enumerate}
  \item Adapt the test scripts you wrote earlier this semester to 
	  enter some sample data into your MongoDB database.
  \item Use MapReduce from within the MongoDB shell to get a count of the number of splatts.
  \item Add a function to your Rails UsersController to return the total number of splatts.

 \end{enumerate}

\end{frame}

\end{document}
