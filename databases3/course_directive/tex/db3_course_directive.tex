\documentclass{article}
\usepackage{graphicx}
\usepackage{wrapfig}
\usepackage{inconsolata}
\usepackage{enumerate}
\usepackage[margin = 2.25cm]{geometry}

\begin{document}

\begin{figure}
\includegraphics[width=30mm]{oplogo.png}
\end{figure}

\title{Course Directive\\IN705 Databases Three\\Semester Two, 2014}
\date{}
\maketitle

\section*{Description}

\section*{Course Information}
\begin{itemize}
  \item 15 Credits
  \item Prerequisites: Databases 2
\end{itemize}

\section*{Lecturer}
\begin{tabular}{lr}

  % after \\: \hline or \cline{col1-col2} \cline{col3-col4} ...
  Tom Clark &    \\
     Office: & D311 \\
     Phone & 470 4356 \\
     Email: & \texttt{tom.clark@op.ac.nz} \\
\end{tabular}

\section*{Course Dates}
\begin{tabular}{lr}
Term 1 (10 weeks) & 21 July - 26 September \\
Mid semester break & 27 September - 12 October \\
Term 2 (6 weeks) & 13 October - 21 November \\
\end{tabular}

\section*{Learning Outcomes}
On completion of this paper you will:
\begin{enumerate}
  \item be able to develop a modern, multi-tier database application;
  \item have experience working with relational and NoSQL databases;
  \item have an understanding of the roles and duties of database administrators.
\end{enumerate}
\newpage

\section*{Resources}
There are a few books that will be useful to you in this paper.  You do not necessarily need to buy all of these books.  It may be sufficient to just go to the library to refer to them from time to time.


\begin{description}
 \item [Churcher, C. \emph{Beginning Database Design} 2nd Ed.] This book provides excellent coverage of relational database design.  You should already have a copy from DB2.
 \item [Redmond, E. and Wilson, J. \emph{Seven Databases in Seven Weeks}] As promised in the title, you will learn about seven distinct types of databases with concrete examples and exercises.
 \item [Sadalage, P. and Fowler, M. \emph{NoSQL Distilled}] You don't need to buy this book, but you should read it at some point.  It's a small book that provides some good context about recent developments in NoSQL.
 \item[Ruby, S., Thomas, D. and Hansson, D. H. \emph{Agile Web Development with Rails 4} P2.0] We will use Ruby on Rails in this paper, and this book will be very helpful if you are not familiar with it. 
 \item[Chacon, S. \emph{Pro Git}] We will use Git for version control in this paper, and Git skills are becoming essential for programming in general.  This book is available as a free ebook download from http://git-scm.com/book, so why wouldn't you get it?  Nearly everything you need to know for now is in the first three chapters.
\end{description}

You will need a (free) GitHub account for this paper, \textbf{and you will need to use it}.  Also, documents for this paper are available online at https://github.com/tclark/op-papers. You should also install Git on any machines on which you paln to do your coding.

You will be given access to an Amazon EC2 instance for your development work in this paper.  This is supported by an AWS in Education Grant award. If you primarily use Windows as your workstation operating system, you may also want to set up a Linux system, virtual or otherwise, for development work.  It is possible to use Windows for this, but it's painfully difficult.

\newpage

\section*{Course Content and Schedule}
This schedule is subject to change based on needs of the class.

\renewcommand{\arraystretch}{1.5}
\begin{tabular}{|l|c|l|l|}
\hline
 Week & Week Start & \multicolumn{1}{c|}{Topics}               &   \\ \hline
 1    & 21 Jul     & Introduction, Development environment     &   \\ \hline
 2    & 28 Jul     & Integration application development       &   \\ \hline
 3    &  4 Aug     & Integration application development       &   \\ \hline
 4    & 11 Aug     & Integration application development       & M1  \\ \hline
 5    & 18 Aug     & Client application development            &   \\ \hline
 6    & 25 Aug     & Client application development            & M2  \\ \hline
 7    &  1 Sep     & Document stores (MongoDB)                 &   \\ \hline
 8    &  8 Sep     & Document stores (MongoDB)                 & M3  \\ \hline
 9    & 15 Sep     & Key-value stores (Riak)                   &   \\ \hline
 10   & 22 Sep     & Key-value stores (Riak)                   & M4  \\ \hline
 H1   & 29 Sep     & Holiday                                   &   \\ \hline
 H2   &  6 Oct     & Holiday                                   &   \\ \hline
 11   & 13 Oct     & Graph databases (Neo4J)                   &   \\ \hline
 12   & 20 Oct     & Graph databases (Neo4J)                   & M5  \\ \hline
 13   & 27 Oct     & DBA topics                                &   \\ \hline
 14   &  3 Nov     & DBA topics                                &   \\ \hline
 15   & 10 Nov     & DBA topics                                &   \\ \hline
 16   & 17 Nov     & Exam and SBA                              &   \\ \hline
\end{tabular}

\section*{Assessment}
There are three assessments in this paper.  During much of the semester you will work on a software development project and you will be assessed on your work at various milestones.
At the end of the semester you will take a theory exam and also perform a skills based assessment of DBA skills.
\vspace{1\baselineskip}

Assessments are weighted as follows: \\
\begin{tabular}{|l|c|}
\hline
Assessment                  &  Weighting \\ \hline
Project Work                &  65\% \\ \hline
Theory Exam                 &  25\% \\ \hline
DBA Skills Based Assessment &  10\% \\ \hline
\end{tabular}

\section*{Criteria for Passing}
You must receive and overall average mark of 50\% or higher to pass this paper.

\section*{Course Requirements and Expectations}
\subsection*{Attendance}
This paper is composed of a mix of lectures and self-paced project work.  Attendence is at your discretion. 
However, you are responsible for keeping up with events that take place in class and completing work on schedule. 

\subsection*{Communication}
Important announcements and discussions about the course, assessments, and scheduling may take place during class sessions.  It is your responsibility to be informed about them.  If you cannot attend a class session, be sure to check with another student.

Your student email is an official communication channel. It is your responsibility to regularly check your student email for important course related material, including changes to class scheduling or assessment details. Not checking will not be accepted as an excuse.

You can manage your email at the Student Hub and download the instructions for forwarding your email at http://www.op.ac.nz/students/student-hub/

\subsection*{Polytechnic Closure}
In the event that the Polytechnic is closed or has a delayed opening because of snow or bad weather, you should not attempt to attend class if it is unsafe to do so. It is possible that your instructor will not be able to attend either, so classes will not physically be meeting. However, this does not become a holiday. Rather, material will be available on the Cisco Academy web site covering the material for classes affected by the closure. You are responsible for any material presented in this manner. Information about closure will be posted on the Otago Polytechnic facebook page https://www.facebook.com/OtagoPoly.

\subsection*{Group Work and Originality}
Students in the Bachelor of Information Technology degree are expected to hand in original work.  Students are encouraged to discuss
assignments with their fellow students.  However, all assignments are to be completed as individual works unless group work is explicitly involved.
Failure to submit your own unique work will be treated as plagiarism.

\subsection*{Referencing}
Appropriate referencing is required for all work.  Referencing standards will be specified by your instructor.

\subsection*{Plagiarism}
Plagiarism is submitting someone else's work as your own.  Plagiarism offences are taken seriously and an
assessment that has been plagiarised may be awarded a zero mark.  A definition of plagiarism is in the Student Handbook,
available online or at the school office.

\subsection*{Submission Requirements}
All assignments are to be submitted by the time, date, and method given when the assignment is issued.

\subsection*{Extensions}
Extensions are only available for unusual circumstances.  These must be applied for, and approved, prior to the submission deadline.

\subsection*{Impairment}
In case of sickness contact your lecturer or year co-ordinator as soon as possible, preferably before the test or
assignment is due.  The policy regarding the granting of a mark that considers impaired performance requires a medical
certificate and a medical practitioners signature on a form. You may should refer to the guide on impaired performance
on the student handbook.

\subsection*{Appeals}
If you are concerned about any aspect of your assessment, please approach the lecturer in the first instance.  We support
an open door policy and aim to resolve issues promptly.  Further support is available from the Programme
Manager and Head of School. Otago Polytechnic has a formal process for academic appeals if necessary.

\subsection*{Other Documents}
Regulatory documents relating this course can be found on the Polytechnic website.




\subsection*{Special Resources and Requirements}
If you have any special needs, whether they relate to the course material, the exercises, the assessment, or anything in the course -
then \textit{please} let your instructor know as soon as possible.

\end{document}
