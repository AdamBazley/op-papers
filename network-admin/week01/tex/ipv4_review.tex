% Beamer slide template prepared by Tom Clark <tom.clark@op.ac.nz>
% Otago Polytechnic
% Dec 2012

\documentclass[10pt]{beamer}
\usetheme{Dunedin}
\usepackage{graphicx}
\usepackage{fancyvrb}

\newcommand\codeHighlight[1]{\textcolor[rgb]{1,0,0}{\textbf{#1}}}

\title{Review of IPv4}

\author[IN715]{Networks Administration}
\institute[Otago Polytechnic]{
  Otago Polytechnic \\
  Dunedin, New Zealand \\
}
\date{}
\begin{document}

%----------- titlepage ----------------------------------------------%
\begin{frame}[plain]
  \titlepage
\end{frame}

%----------- slide --------------------------------------------------%
\begin{frame}
  \frametitle{Remember this?}

 \begin{itemize}
  \item An IPv4 address is 32 bits long - 4 bytes
  \item Each byte has 256 possible values (0 - 255)
  \item We usually respresent them in dotted decimal notation 
        \begin{itemize}
          \item 10.50.1.80
          \item 74.125.237.207
          \item 224.0.0.118
        \end{itemize}
  \end{itemize}

\end{frame}


%----------- slide --------------------------------------------------%
\begin{frame}
  \frametitle{Network and host bits}

 \begin{itemize}
  \item Any IP address can be divided into two parts:
        \begin{enumerate}
          \item network
          \item host
        \end{enumerate}
  
 \end{itemize}

\end{frame}


%----------- slide --------------------------------------------------%
\begin{frame}
  \frametitle{Network masks}
 \begin{itemize}
  \item We can identify the network and host bits by examinng the network mask.
  \item Example: 255.255.192.0 \\
        In binary:  11111111.11111111.11000000.00000000 \\
        The 1's indicate network bits and the 0's indicate host bits
  \item We can indicate the same thing by writing /18 - indicating 18 network bits.
 \end{itemize}

\end{frame}


%----------- slide --------------------------------------------------%
\begin{frame}
  \frametitle{Address classes}
  In the absense of a network mask, we can infer it from the address class

 \begin{tabular}{|l|l|l|l|l|}
  \hline
  Class & Leading Octet & Mask & Networks  & Hosts \\ \hline
  A     & 1 - 127       & /8   & 127       & 16,777,216  \\ \hline
  B     & 128 - 192     & /16  & 16,384    & 65,536  \\ \hline
  C     & 192 - 223     & /24  & 2,097,152 &  256 \\ \hline
 
 \end{tabular}

\end{frame}


%----------- slide --------------------------------------------------%
\begin{frame}
  \frametitle{Subnetting}

 \begin{itemize}
  \item Given an IPv4 network, we can divide it into smaller subnetworks, or subnets.
  \item We do this by ``borrowing" host bits and adding them to the network portion of the address.
 \end{itemize}

\end{frame}


%----------- slide --------------------------------------------------%
\begin{frame}
  \frametitle{Subnetting example}

 \begin{itemize}
  \item Given 192.168.1.0/24
  \item We ``borrow" 2 host bits to create 4 subnets: \\
        192.168.1.0/26 \\
        192.168.1.64/26 \\
        192.168.1.128/26 \\
        192.168.1.192/26 
 \end{itemize}

\end{frame}


%----------- slide --------------------------------------------------%
\begin{frame}
  \frametitle{Private networks}
   
   Some address ranges can be used for private networks. These addresses
   are not publically routable.

 \begin{itemize}
  \item 10.0.0.0/8  
  \item 172.16.0.0/16 - 172.31.0.0/16 
  \item 192.168.0.0/24 - 192.168.255.0/24
 \end{itemize}
 Network address translation (NAT) can be used to allow privately addressed hosts to connect to the internet.

\end{frame}



%----------- slide --------------------------------------------------%
\begin{frame}
  \frametitle{Network addresses}

 \begin{itemize}
  \item An address like 192.168.10.0/24 is usually a \emph{network address}. 
  \item Network addresses do not refer to any one host.  They refer to entire networks in aggregate.
 \end{itemize}

\end{frame}


%----------- slide --------------------------------------------------%
\begin{frame}
  \frametitle{Broadcast addresses}

 \begin{itemize}
  \item An address like 192.168.10.255 is usually a \emph{broadcast address}.
  \item Broadcast addresses do not refer to any one host.  
  \item A packet sent to a broadcast address is intended to be received by \textbf{every} host on a network. 
 \end{itemize}
\end{frame}

%----------- slide --------------------------------------------------%
\begin{frame}
  \frametitle{Gateway addresses}

 \begin{itemize}
  \item Hosts on a network are usually configured with a \emph{gateway address} or \emph{default gateway}. These are the addresses of local router interfaces.
  \item These are ordinary host addresses on the network. Unlike network or broadcast addresses, you can't recognise a gateway address just by looking at it.
  \item Packets whose destinations are off the local network must be forwarded through the gateway address.
 \end{itemize}



\end{frame}

%----------- slide --------------------------------------------------%
\begin{frame}
	\frametitle{Ports}
	
	\begin{itemize}
	    \item IP adresses can only identify hosts on a network.
	    \item One host may be running many processes that receive network packets.
		\item Transport layer segments use \emph{port numbers} to identify the process that should
		          receive the data.
		\item Common network services use standard port numbers, like 80 for web, 22 for ssh, and 53 for DNS.
	\end{itemize}
	
	
	
\end{frame}

\end{document}
