% Beamer slide template prepared by Tom Clark <tom.clark@op.ac.nz>
% Otago Polytechnic
% Dec 2012

\documentclass[10pt]{beamer}
\usetheme{CambridgeUS}
\usepackage{graphicx}
\usepackage{fancyvrb}


\title{Introdution to Firewalls}

\author[IN715]{Networks Three}
\institute[Otago Polytechnic]{
  Otago Polytechnic \\
  Dunedin, New Zealand \\
}
\date{}
\begin{document}

%----------- titlepage ----------------------------------------------%
\begin{frame}[plain]
  \titlepage
\end{frame}


%----------- slide --------------------------------------------------%
\begin{frame}
  \frametitle{The Internet is a pretty scary place}

 \begin{itemize}
  \item An unprotected system exposed to the Internet will be subect to attacks within minutes.
  \item Hosts on our networks typically emit network traffic that should not be visible outside 
	  our networks.
  \item Compromised machines may send unwanted traffic that should be contained.
 \end{itemize}
 
 Conclusion:  We need firewalls to control the flow of network traffic.
\end{frame}


%----------- slide --------------------------------------------------%
\begin{frame}
  \frametitle{Host or network based}

 \begin{itemize}
	 \item Most operating systems include firewalling capabilities to protect individual hosts.
	 \item Network firewalls may be deployed at network perimeters to protect entire networks.
	 \item A comprehensive security strategy should include both.
 \end{itemize}

\end{frame}


%----------- slide --------------------------------------------------%
\begin{frame}
  \frametitle{Application firewalls}

 Application firewalls work at the application layers, inspecting the payload data
 for unwanted traffic.

 Examples:
 \begin{itemize}
	 \item Email spam and virus filters
	 \item Web filters
 \end{itemize}

\end{frame}



%----------- slide --------------------------------------------------%
\begin{frame}
  \frametitle{Packet filters}

 Packet filters inspect individual packets for network and transport layer information.  They 
 pass or block traffic according to rules based on

 \begin{itemize}
	 \item Source and destination IP addresses
	 \item Source and destination ports
	 \item Transport layer protocols (TCP, UDP, ICMP, ICMP6)
	 \item Traffic direction (inbound or outbound)
	 \item Connection state
 \end{itemize}

\end{frame}


%----------- slide --------------------------------------------------%
\begin{frame}
  \frametitle{PF: OpenBSD packet filter}

 \begin{itemize}
   \item PF (Packet Filter) is the firewall package included in OpenBSD.
   \item It is installed and enabled by default (It's just configured to
	   pass all traffic.
   \item It is configured using the file \texttt{/etc/pf.conf} and from the 
	   commmand line using \texttt{pfctl}
 \end{itemize}

\end{frame}



%----------- slide --------------------------------------------------%
\begin{frame}[fragile]
	\frametitle{Some handy \texttt{pfctl} commands}

 \begin{verbatim}
 # pfctl -f /etc/pf.conf     Load the pf.conf file
 # pfctl -nf /etc/pf.conf    Parse the file, but don't load it
 # pfctl -sr                 Show the current ruleset
 # pfctl -ss                 Show the current state table
 # pfctl -si                 Show filter stats and counters
 # pfctl -sa                 Show EVERYTHING it can show
 \end{verbatim}

\end{frame}


%----------- slide --------------------------------------------------%
\begin{frame}[fragile]
\frametitle{PF rules}

 PF inspects packets according to its set of \emph{rules}.  When a packet matches a rule's
 selection criteria, the rule's action may be carried out.

 \begin{verbatim}
   block in all
   pass in from all to 10.4.0.3 22
   pass out from 192.160.1.0/24 to any port www
 \end{verbatim}

\end{frame}


%----------- slide --------------------------------------------------%
\begin{frame}[fragile]
  \frametitle{Rule syntax}

 \begin{verbatim}
action [direction] [log] [quick] [on interface] [af] 
  [proto protocol] [from src_addr [port src_port]] 
  [to dst_addr [port dst_port]] [flags tcp_flags] 
  [state]
 \end{verbatim}

\end{frame}

%----------- slide --------------------------------------------------%
\begin{frame}
  \frametitle{Rule order}

 \begin{itemize}
	 \item Rules are processed in order.
	 \item A packet may match many rules.  
	 \item The last rule matched wins.
	 \item We can short-circuit this with the \texttt{quick} option
 \end{itemize}
\end{frame}

%----------- slide --------------------------------------------------%
\begin{frame}
  \frametitle{More information}

 \url{http://www.openbsd.org/faq/pf/index.html}
   



\end{frame}


\end{document}
