\documentclass{article}
\usepackage{graphicx}
\usepackage{wrapfig}

\usepackage{verbatim}
\usepackage[parfill]{parskip}
\usepackage[margin = 2.5cm]{geometry}

\usepackage[T1]{fontenc}


\begin{document}

\title{ Updating DHCP \\ IN715 Networks Administration}
\date{\today}
\maketitle

\section*{Introduction}
We saw last week that we would need to update our \texttt{dhcpd} to a newer version in order to use the newer failover and load balancing features.  In this lab we wiil make some changes to our network configuration to allow us to upgrade.

\section{Configure NAT}
We can't install a new version of \texttt{dhcpd} until we can connect our server to the Internet.  But since we are using a private address space\footnote{RFC 1918} we must configure \emph{Network Address Translation} (NAT) to do this.

NAT will

\begin{enumerate}
	\item Track the state of outgoing packets;
	\item Modify the source address of packets leaving our network;
	\item Modify the destination address of response packets so that they can be routed correctly. 
\end{enumerate}

To track and modify packets we need to configure OpenBSD's \texttt{pf} firewall.  We need to add two rules to \texttt{/etc/pf.conf} on \texttt{router1} to get it to perform NAT.  Remember that our outside interface is \texttt{em0}, and suppose the IP address on that interface is 10.25.1.100.  Add the following lines to the end of the file

\begin{verbatim}
match out on em0 \
    from 172.16.5.0/24  \
    nat-to 10.25.1.100

pass out on em0 \
    from 10.25.1.100
...
\end{verbatim}

The first rule performs the NAT and the second allows the NAT'ed packets to exit the network.  Load your new firewall rules into the system with the command

\texttt{pfctl -f /etc/pf.conf}

\section{Update your dhcpd}
We can use OpenBSD's package system to install the new version of \texttt{dhcpd} with the command

\texttt{pkg\_add isc-dhcp-server-4.2.5.1p0}  

This will install the new \texttt{dhcpd} in \texttt{/usr/local/sbin}.  Now you just need to modify the init script at \texttt{/etc/rc.d/dhcpd}.  Change the ``\texttt{daemon}'' value to \texttt{/usr/local/sbin/dhcpd}.

Now you can use failover peers in your \texttt{dhcpd.conf} as we discussed last week.  N.B.: Be sure your failover peer configuration appears before the subnet declarations in your file, like this:

\begin{verbatim}
failover peer "dhcp-failover" {
    # relevant config ...
}

subnet 172.16.5.0 netmask 255.255.255.0 {
    pool {
        range 172.16.5.100 172.16.5.150;
        failover peer "dhcp-failover";
        }
        ...
}
\end{verbatim}

If you try to use the failover peer before it is defined in the file you will get errors when attempting to start \texttt{dhcpd}.

\end{document}
