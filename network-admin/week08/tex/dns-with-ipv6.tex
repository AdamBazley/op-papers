% Beamer slide template prepared by Tom Clark <tom.clark@op.ac.nz>
% Otago Polytechnic
% Dec 2012

\documentclass[10pt]{beamer}
\usetheme{Dunedin}
\usepackage{graphicx}
\usepackage{fancyvrb}

\newcommand\codeHighlight[1]{\textcolor[rgb]{1,0,0}{\textbf{#1}}}

\title{DNS with IPv6}

\author[IN715]{Networks Three}
\institute[Otago Polytechnic]{
  Otago Polytechnic \\
  Dunedin, New Zealand \\
}
\date{}
\begin{document}

%----------- titlepage ----------------------------------------------%
\begin{frame}[plain]
  \titlepage
\end{frame}


%----------- slide --------------------------------------------------%
\begin{frame}
  \frametitle{IPv6 works with DNS}

 \begin{itemize}
  \item It is possible to integrate IPv6 DNS resources together with
        existing IPv4 ones.
  \item Only one new resource record type is required.
  \item Reverse zones work in a similar way.
  \item Recent versions of the main DNS servers are compatible with IPv6.
 \end{itemize}
\end{frame}
\begin{frame}
  \frametitle{Initial Considerations}

 \begin{itemize}
  \item Your DNS servers need IPv6 connectivity.
  \item Your domain name registrar needs to support IPv6.\footnote{See http://dnc.org.nz/content/srs\_registrar\_list.html}
  \item Your TLD needs to support IPv6. (.nz does)
 \end{itemize}
\end{frame}
\begin{frame}
  \frametitle{Configuring DNS}

 \begin{itemize}
  \item Ensure that your NS records point to servers with IPv6 addresses as well as IPv4 addresses.
  \item Configure your DNS servers to listen on IPv6 interfaces, e.g.: \
          \texttt{ listen-on-v6 \{ any; \}} in named.conf.
 \end{itemize}
\end{frame}

\begin{frame}[fragile]
  \frametitle{Forward lookup: AAAA}
  The IPv6 equivalent of an A record is an AAAA record:

  \begin{verbatim}
   gredo  IN  A  2402:6000:0:101::83cb:30a
  \end{verbatim}

For machines that have both IPv4 and IPv6 addresses, be sure that you 
have distinct A and AAAA records for them, e.g.  \texttt{gredo.ip4} 
and \texttt{gredo.ip6}.
\end{frame}
\begin{frame}
  \frametitle{Reverse lookup}
  Reverse lookup is a wee bit more complicated.  for example, suppose our
  network is \texttt{2400:330A:0:118/64}
  
  What is the reverse zone?

 
\end{frame}
\begin{frame}
  \frametitle{2400:330A:0:118/64}
  \begin{enumerate}
    \item Expand the network address:  \texttt{2400:330A:0000:0118}
    \item Reverse the characters and put dots between them
          \texttt{8.1.1.0.0.0.0.0.A.0.3.3.0.0.2.4}
    \item Append \texttt{.ip6.arpa}
  \end{enumerate}

  So the zone is 8.1.1.0.0.0.0.0.A.0.3.3.0.0.2.4.ip6.arpa.
 
\end{frame}
\begin{frame}[fragile]
  \frametitle{PTR records}
  PTR records work the same way in IPv6. Since the resulting names get
  very long, so you want to abbreviate.
  \begin{verbatim}
   
a.0.3.0.b.c.3.8.0.0.0.0.0.0.0.0    IN    PTR    gredo.example.com.
  \end{verbatim}


There are some online tools that will help you generate these entries, 
e.g.: http://rdns6.com/zone.
\end{frame}
\end{document}
