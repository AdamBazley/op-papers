\documentclass{article}
\usepackage{graphicx}
\usepackage{enumerate}
\usepackage{verbatim}
\usepackage[parfill]{parskip}
\usepackage[margin = 2.5cm]{geometry}

\usepackage[T1]{fontenc}


\begin{document}

\title{ Lab 7.2 DNS on Windows \\ IN715 Networks Three}
\date{\today}
\maketitle

\section*{Introduction}
A properly functioning DNS server setup includes at least two DNS servers to provide a necessary degree of redundancy.  In this lab we will set up DNS on our Windows 2012 servers to do this. We will configure our Windows server to act as a \emph{slave}, meaning that it will get its zone information from our primary BIND server.

\section{Install the DNS server role}
Open your Server Manager and select the ``Add Roles and Features'' option.  Choose a ``Role of feature based installation'' and click Next.  Continue clicking through the options, selcting your server from the server pool (it should be the only choice) and selecting the DNS server role.  Once the installation is complete your DNS server will be ready to configure and you should have a lovely DNS tile on your start screen.

\section{Configure DNS}
Start your DNS Server Manager by clicking the DNS tile on your start screen.  Find ``Forward DNS Zone'' in the left side menu, right click it and select ``New Zone''.  Work through the wizard, slecting the Secondary Zone option and supplying the address of your primary DNS server.  The DNS Manager will verify that the master server is funtioning properly as part of the setup.

Follow the same procedure to add your reverse lookup zones.
\end{document}
