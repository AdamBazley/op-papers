% Beamer slide template prepared by Tom Clark <tom.clark@op.ac.nz>
% Otago Polytechnic
% Dec 2012

\documentclass[10pt]{beamer}
\usetheme{CambridgeUS}
\usepackage{graphicx}
\usepackage{fancyvrb}

\newcommand\codeHighlight[1]{\textcolor[rgb]{1,0,0}{\textbf{#1}}}

\title{Serving DHCP from OpenBSD}

\author[IN715]{Networks Three}
\institute[Otago Polytechnic]{
  Otago Polytechnic \\
  Dunedin, New Zealand \\
}
\date{}
\begin{document}

%----------- titlepage ----------------------------------------------%
\begin{frame}[plain]
  \titlepage
\end{frame}


%----------- frame  ----------------------------------------------%
\begin{frame}
  \frametitle{Step 1: Plan}

  We need to answer the following questions:
  \begin{itemize}
    \item What is our overall network topology?
    \item On what networks will we serve DHCP?
    \item What are our service level requirements?
    \item For each network
      \begin{itemize}
        \item How many clients will we serve?
        \item What is their usage profile?
        \item Are there any machines that need static addresses?
      \end{itemize}
  \end{itemize}  
\end{frame}


%----------- frame  ----------------------------------------------%
\begin{frame}
  \frametitle{Planning, cont.}
  
  We also need specific information to serve to clients
  \begin{itemize}
    \item The location's domain name
    \item Client IP address ranges
    \item DNS server addresses
    \item Gateway addresses
    \item Subnet masks, broadcast addresses
    \item Possibly others
  \end{itemize}  
\end{frame}


%----------- frame  ----------------------------------------------%
\begin{frame}
  \frametitle{OpenBSD server setup}
  \begin{itemize}
    \item The ISC DHCP server, \emph{dhcpd}, is installed by default.
    \item The configuration file is \texttt{/etc/dhcpd.conf}
  \end{itemize}  
\end{frame}

%----------- frame  ----------------------------------------------%
\begin{frame}
  \frametitle{A simple scenario}
  Suppose we have an OpenBSD server 
  \begin{itemize}
    \item connected to an external interface on 10.25.0.0/16
    \item connected to an internal interface on 192.168.1.0/24
    \item we want to serve DHCP on to clients on the internal network
  \end{itemize}  
  
  What goes in \texttt{dhcpd.conf}?
\end{frame}


%----------- frame  ----------------------------------------------%
\begin{frame}
  \frametitle{Global options}
  The following lines specify global options that apply to every network
  we serve, unless overridden later.
\end{frame}


%----------- frame  ----------------------------------------------%
\begin{frame}
  \frametitle{Domain name}
  
  \texttt{ option domain-name "op.ac.nz"; }

  This tells clients to append our domain name to unqualified hostnames, 
  i.e.  \texttt{ssh foo} goes to \texttt{foo.op.ac.nz}.
\end{frame}


%----------- frame  ----------------------------------------------%
\begin{frame}
  \frametitle{DNS servers}

  \texttt{ option domain-name-servers 10.50.1.80, 10.50.1.82; }

  Tell the clients to use these DNS servers.

\end{frame}


%----------- frame  ----------------------------------------------%
\begin{frame}
  \frametitle{Lease times}

  \texttt{ default-lease-time 86400 }
  
  \texttt{ max-lease-time 259200 }

  By default, our clients get a one day lease.  They can request a longer one,
  and we will allow up to three days.
  

\end{frame}

%----------- frame  ----------------------------------------------%
\begin{frame}
  \frametitle{Authoritative}
  
  \texttt{ authoritative; }
 
  Identifies our server as authoritative.

\end{frame}

%----------- frame  ----------------------------------------------%
\begin{frame}
  \frametitle{Subnetworks}
 
  Next, we create a configuration block for every subnet.  There are subnet-specific options, and we can override global options.
\end{frame}


%----------- frame  ----------------------------------------------%
\begin{frame}[fragile]
  \frametitle{External network}
  \begin{verbatim}
    subnet 10.25.0.0 netmask 255.255.0.0 {
    }
  \end{verbatim}  

  Even though we won't serve DHCP on this subnet, it's good practice to 
  create an empty block for it.  Since it's directly connected to the server,
  dhcpd should know about it.

\end{frame}


%----------- frame  ----------------------------------------------%
\begin{frame}[fragile]
  \frametitle{Internal network}
  \begin{verbatim}
    subnet 192.168.1.0 netmask 255.255.255.0 {
        range 192.168.1.100 - 192.168.1.200;
        option routers 192.168.1.1;
        option broadcast-address 192.168.1.255;
    }
  \end{verbatim}  

  Even though we won't serve DHCP on this subnet, it's good practice to 
  create an empty block for it.  Since it's directly connected to the server,
  dhcpd should know about it.

\end{frame}

%----------- frame  ----------------------------------------------%
\begin{frame}[fragile]
  \frametitle{Internal network with a static host}
  \begin{verbatim}
   shared-network office {
    option routers 192.168.1.1;
    option broadcast-address 192.168.1.255;
    subnet 192.168.1.0 netmask 255.255.255.0 {
        range 192.168.1.100 - 192.168.1.200;
    }

   host bob {
     hardware-ethernet 00:A0:78:6E:8E:A1;
     fixed-address 192.168.1.10;
   }
  }
  \end{verbatim}  

  The host \texttt{bob} gets a fixed address based on the MAC address.  Putting
  the subnet and the host declaration inside the \texttt{shared-network} block
  means that all clients in the network pick up the common \texttt{routers} and  \texttt{broadcast-address} options.  

\end{frame}

\end{document}
