% Beamer slide template prepared by Tom Clark <tom.clark@op.ac.nz>
% Otago Polytechnic
% Dec 2012

\documentclass[10pt]{beamer}
\usetheme{CambridgeUS}
\usepackage{graphicx}
\usepackage{fancyvrb}


\title{Lists, Tables, and  Macros}

\author[IN715]{Networks Three}
\institute[Otago Polytechnic]{
  Otago Polytechnic \\
  Dunedin, New Zealand \\
}
\date{}
\begin{document}

%----------- titlepage ----------------------------------------------%
\begin{frame}[plain]
  \titlepage
\end{frame}


%----------- slide --------------------------------------------------%
\begin{frame}
  \frametitle{Firewall rules}
 Last time we started creating firewall rules. They include
 \begin{itemize}
    \item an action (\texttt{block} or \texttt{drop};
    \item specifications for matching packets, such as
        \begin{itemize}
             \item source or destination IP address
             \item source or destination port
             \item transport protocol
             \item network interface
        \end{itemize}
    \item Example: \texttt{pass in on em0 to 172.20.44.8 to port 80} 
 \end{itemize}

\end{frame}


%----------- slide --------------------------------------------------%
\begin{frame}[fragile]
  \frametitle{Multiple rules}

 In many situations you may find yourself writing pretty repetitive rules.

 \begin{verbatim}
 pass in on em0 proto tcp to 192.168.5.5 port 22    
 pass in on em0 proto tcp to 192.168.5.5 port 25    
 pass in on em0 proto tcp to 192.168.5.5 port 80    
 pass in on em0 proto tcp to 192.168.5.5 port 443   
 \end{verbatim}
 
 You can condense a rule set like this with a list.
 \begin{verbatim}
 pass in on em0 proto tcp to 192.168.5.5 port { 22 25 80 443 }   
 \end{verbatim}
 PF just expands your rule to four rules as they are shown above.
\end{frame}



%----------- slide --------------------------------------------------%
\begin{frame}[fragile]
  \frametitle{List uses}

 You can use a list anywhere where you want to insert mulitple values in a 
 rule set.

 \begin{verbatim}
 block out from { 192.168.6.211 10.115.8.20 } port www

 pass in proto { udp tcp } to 172.18.18.6 port 53   
 \end{verbatim}

\end{frame}


%----------- slide --------------------------------------------------%
\begin{frame}[fragile]
  \frametitle{List expansion}
 Rules with lists are expanded out to multiple rules.  You need
 to be careful to avoid unintended consequences.
 \begin{verbatim}
   pass in on fxp0 from { 10.0.0.0/8, !10.1.2.3 }
 \end{verbatim}
 Expands to 
 \begin{verbatim}
   pass in on fxp0 from 10.0.0.0/8
   pass in on fxp0 from !10.1.2.3
 \end{verbatim}
 Which is almost certainly not what was intended.
\end{frame}


%----------- slide --------------------------------------------------%
\begin{frame}[fragile]
  \frametitle{Macros}

 Macros are user-defined variables.
 \begin{verbatim}
   allowed_ports = "{ 22 80 443 }"
   web_server = "10.10.1.5"
   pass in to $web_server port $allowed_ports
 \end{verbatim}

\end{frame}


%----------- slide --------------------------------------------------%
\begin{frame}[fragile]
  \frametitle{Macros in lists}

 You can use macros in lists.
 \begin{verbatim}
   malicious_ips = "{ 192.168.15.7 172.21.233.18 }"
   marketing = "10.40.1.0/24"
   block all from { $malicious_ips $marketing }
 \end{verbatim}

\end{frame}


%----------- slide --------------------------------------------------%
\begin{frame}
  \frametitle{Tables}

 \begin{itemize}
   \item Tables hold groups of IP addresses.
   \item Tables lookup are faster and use less resources than lists.
   \item Tables can be modified on the fly.
   \item Table syntax handles negations nicely (unlike lists).
 \end{itemize}

\end{frame}


%----------- slide --------------------------------------------------%
\begin{frame}[fragile]
  \frametitle{Declaring tables}

 \begin{verbatim}
   table <office_net> { 192.168.10.0/24 }
   table <allowed_out> const { 10.10.0.0/16 !10.10.5.5 }
   table <spammers> persist file "/etc/spammers"
 \end{verbatim}

\end{frame}


%----------- slide --------------------------------------------------%
\begin{frame}[fragile]
  \frametitle{Manipulating tables}

 Tables not declared as \texttt{const} can be modified with \texttt{pfctl}
 \begin{verbatim}
   pfctl -t spammers -T add 203.0.113.0/24

   pfctl -t spammers -T show

   pfctl -t spammers -T delete 203.0.113.0/24
 \end{verbatim}

\end{frame}

\end{document}
