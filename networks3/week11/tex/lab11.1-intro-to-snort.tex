\documentclass{article}
\usepackage{graphicx}
\usepackage{enumerate}
\usepackage{verbatim}
\usepackage{hyperref}
\usepackage[parfill]{parskip}
\usepackage[margin = 2.5cm]{geometry}

\usepackage[T1]{fontenc}


\begin{document}

\title{ Lab 11.1 Introduction to Snort\\ IN715 Networks Three}
\date{\today}
\maketitle

\section*{Introduction}
Snort is a powerful network traffic analysis tool.  It has the capacity to 

\begin{itemize}
		 \item sniff network packets in real time;
		 \item log packet data to a file;
		 \item analyse captured packets for intrusion dectection.
	 \end{itemize}

Ultimately it is the last item that is most interesting to us.

Carry out this lab on your \texttt{router2} system.

\section{Install and try Snort}
Installation on our OpenBSD servers is extremely easy.  Just enter the command

\begin{verbatim}
  pkg_add snort
\end{verbatim}

and it will be installed.

Try out Snort by having it sniff packets and output its data to the console with the following command.

\begin{verbatim}
snort -v --daq-dir/usr/local/lib/daq
\end{verbatim}

Information about packets will be displayed in the console. Carry out some network activity to generate some interesting packets to log.  Enter \texttt{Ctrl-c} to stop Snort when you are done.  If you want to see the packet payload data as well, invoke Snort with the -d option:


\begin{verbatim}
snort -dv --daq-dir/usr/local/lib/daq
\end{verbatim}

\section{Logging Snort data to a file}
In a typical network setting Snort will capture too much data to be usefull examined in standard output.  It is more productive to have Snort write its data to a logfilethat we can examine later.  We can do this by using Snort's \texttt{-l} option, like this:


\begin{verbatim}
mkdir log
snort -dv -l ./log --daq-dir/usr/local/lib/daq
\end{verbatim}

Snort will then write its data into a file in the \texttt{log} directory.  This logfile is in a binary format that can be read with tools like \texttt{tcpdump} or with snort itself.  After having Snort log for a few minutes, stop Snort by entering \texttt{Ctrl-c}.  Then you can inspect the logged data by converting it to a text file with a command like this:


\begin{verbatim}
snort -dv --daq-dir/usr/local/lib/daq -r ./log/<log file name>   > snort_log
\end{verbatim}

Now the file \texttt{snort\_log} contains the data in text format.

\section{Running the Snort service}
To use Snort as a proper \emph{Network Intrusion Detection System} (NIDS), we need to install some processing rules that identify packets of interest.  Rather than log every packet seen, we want to indentify those that match a suspected attack signature.  We can write our own rules, but it's best to start with a standard set.

First, make a directory to store our rules:

\begin{verbatim}
  mkdir /usr/local/snort
  cd /usr/local/snort
  \end{verbatim}

Now download some rules an unpack them.

\begin{verbatim}
pkg_add curl
curl http://kate.ict.op.ac.nz/~tclark/snortrules.tgz > snortrules.tgz
tar -xzf snortrules.tgz
\end{verbatim}

Now you need to modify the configuration in \texttt{/etc/snort/snort.conf} Modify the \texttt{RULE\_PATH}, \texttt{SO\_RULE\_PATH}, and \texttt{PREPROC\_RULE\_PATH} variables to point to your directories in \texttt{usr/local/snort}.

You also need to specify an output format.  In the config file, look for a line that starts \texttt{output unified2} and uncomment that.

Now you can start Snort with the command \texttt{/etc/rc.d/snort start}.  Log files should be created under \texttt{/var/snort}.  Next time we will see how we can use those logs.

\end{document}
