% Beamer slide template prepared by Tom Clark <tom.clark@op.ac.nz>
% Otago Polytechnic
% Dec 2012

\documentclass[10pt]{beamer}
\usetheme{CambridgeUS}
\usepackage{graphicx}
\usepackage{fancyvrb}

\newcommand\codeHighlight[1]{\textcolor[rgb]{1,0,0}{\textbf{#1}}}

\title{DNS Zones}

\author[IN715]{Networks Three}
\institute[Otago Polytechnic]{
  Otago Polytechnic \\
  Dunedin, New Zealand \\
}
\date{}
\begin{document}

%----------- titlepage ----------------------------------------------%
\begin{frame}[plain]
  \titlepage
\end{frame}



%----------- slide --------------------------------------------------%
\begin{frame}
  \frametitle{DNS Zones}

 \begin{itemize}
  \item Recall that last time we saw how the DNS hierarchy can be viewed as a tree.
  \item Each node on that tree is a \emph{DNS Zone}.
  \item A zone is composed of a set of \emph{Resource Records} of various types.
  \item Information about a particular zone is kept in a \emph{Zone File}.  These files conform to a standard format\footnote{RFC 1034 and RFC 1035}.
 \end{itemize}

\end{frame}


%----------- slide --------------------------------------------------%
\begin{frame}
  \frametitle{Zone file names}

 \begin{itemize}
  \item There is no particular requirement for how zone files are named.
  \item Best practice is to indicate the domain name for the zone in the filename, e.g., \texttt{db.example.com} for the \emph{example.com} zone.
 \end{itemize}

\end{frame}


%----------- slide --------------------------------------------------%
\begin{frame}
  \frametitle{Time To Live (TTL)}

 \begin{itemize}
  \item We start a zone (in BIND 9) by specifying its \emph{TTL}, e.g.,
      \begin{itemize}
        \item \texttt{\$TTL 3h}
        \item \texttt{\$TTL 1d}
        \item \texttt{\$TTL 1w}
      \end{itemize}
  \item The TTL specifies the length of time for which our zone data should be cached.
  \item A high TTL saves load on our servers, but it means that changes will take more time to propagate.
 \end{itemize}
\end{frame}



%----------- slide --------------------------------------------------%
\begin{frame}[fragile]
  \frametitle{Statement of Authority (SOA)}
 The SOA states that this server is an \emph{authoritative} source of informatiion about our zone.
 \begin{verbatim}
  example.com. IN SOA ns1.example.com. tech.somedomain.net. (
       20140821092215	; serial number
       3h               ; slave refresh 
       1h               ; slave retry
       3d               ; slave expires
       1h )             ; negative ttl
 \end{verbatim}
\end{frame}


%----------- slide --------------------------------------------------%
\begin{frame}[fragile]
  \frametitle{Nameserver (NS) records}
 NS records identify the authoritative name servers for our zone
 \begin{verbatim}
   example.com.  IN NS ns1.example.com.
   example.com.  IN NS ns2.example.com.
   example.com.  IN NS ns.otherdomain.com.
 \end{verbatim}

\end{frame}


%----------- slide --------------------------------------------------%
\begin{frame}[fragile]
  \frametitle{Address (A) records}
 A records map host names to IP addresses.
 \begin{verbatim}
   fred.example.com. IN A  123.220.44.91

   ;a host with two addresses
   barney.example.com. IN A 71.44.116.17
   barney.example.com. IN A 123.211.16.100

   ;A records can point to the same address as other A records
   ws1.example.com. IN A 71.44.116.17
   ws2.example.com. IN A 123.211.16.100
 \end{verbatim}

\end{frame}


%----------- slide --------------------------------------------------%
\begin{frame}[fragile]
  \frametitle{Alias (CNAME) records}

 A CNAME record creates an alias for another hostname
 \begin{verbatim}
   dino.example.com. IN CNAME fred.example.com. 
 \end{verbatim}

\end{frame}


%----------- slide --------------------------------------------------%
\begin{frame}[fragile]
  \frametitle{Mail Exchange (MX) records}
 MX records identify servers that receive mail for our domain
 \begin{verbatim}
   example.com. IN MX 10  wilma.example.com.  
   example.com. IN MX 20  betty.bedrock.org.
 \end{verbatim}

\end{frame}


%----------- slide --------------------------------------------------%
\begin{frame}[fragile]
  \frametitle{Reverse zone files}

 Recall that we use the in-addr.arpa domain to support reverse DNS 
 lookups.  This requires another zone file
 
 File: db.192.168.10
 \begin{verbatim}
   $TTL 3h
   10.168.192.in-addr.arpa. IN SOA ns2.example.com. tec.sdn.net (
    ... SOA stuff ... )
       10.168.192.in-addr.arpa.  IN NS ns1.example.com.
       10.168.192.in-addr.arpa.  IN NS ns2.example.com.

       1.10.168.192.in-addr.arpa.  IN PTR pebbles.example.com.
       2.10.168.192.in-addr.arpa.  IN PTR slate.example.com.
       41.10.168.192.in-addr.arpa. IN PTR bambam.rubble.com.


   
 \end{verbatim}

\end{frame}




%----------- slide --------------------------------------------------%
\begin{frame}[fragile]
  \frametitle{}

 \begin{verbatim}
   
 \end{verbatim}

\end{frame}

%----------- slide --------------------------------------------------%
\begin{frame}[fragile]
  \frametitle{Abbreviations}
 There are a few ways we can abbreviate resource records.

 1.  @ Notation: If the domain name in a record is the same as the 
 \emph{origin}, it can be abbreviated with an @.
 \begin{verbatim}
  example.com. IN SOA ns1.example.com. tech.somedomain.net. (
 
   is the same as 
 
  @ IN SOA ns1.example.com. tech.somedomain.net. (


   
 \end{verbatim}

\end{frame}

%----------- slide --------------------------------------------------%
\begin{frame}[fragile]
  \frametitle{Appending domain names}

 You can leave the domain names off of the record names when they match the origin.

 \begin{verbatim}
   fred.example.com. IN A  123.220.44.91

   becomes

   fred IN A  123.220.44.91


   1.10.168.192.in-addr.arpa.  IN PTR pebbles.example.com.

   becomes

   1  IN PTR pebbles.example.com.
   
 \end{verbatim}

\end{frame}



%----------- slide --------------------------------------------------%
\begin{frame}[fragile]
  \frametitle{Repeated names}

 You can omit repeated resource names used on multiple records.
 \begin{verbatim}
   
   barney.example.com. IN A 71.44.116.17
                       IN A 123.211.16.100
 
\end{verbatim}

\end{frame}


\end{document}
