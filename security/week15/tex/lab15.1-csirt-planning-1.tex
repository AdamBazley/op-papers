\documentclass{article}
\usepackage{enumerate}
\usepackage{verbatim}
\usepackage{hyperref}
\usepackage[parfill]{parskip}
\usepackage[margin = 2.5cm]{geometry}

\usepackage[T1]{fontenc}


\begin{document}

\title{ Lab 15.1: CSIRT \\ IN618 Security}
\maketitle

\section*{Introduction}
In last week's reading we learned about the need to organise \emph{Computer Security Incident Response Teams} (CSIRTs) and to prepare incident response plans before security incidents occur.  In this lab we will work in small groups to start preparing plans to respond to various incidents. 

\section{Instructions}
In your groups, discuss responses to the questions below. At the end of the class session you will briefly present your responses to the rest of the class. Each person on the team will then individually write a document with an initial response plan for your incident type and submit it in class on Thursday.

\section{Questions to consider}
In your CSIRT plan you should answer the following questions.

\begin{enumerate}
	\item Considering the example response priorities on the session 14.1 slides, what are the appropriate priorities for the CSIRT in your scenario?
	
	\item What are some ways that people in your organisation might initially recognise this type of security incident?  How should they report their concerns?
	
	\item Who should perform the initial assessment of the incident and what are some steps they may take to perform it?
	
	\item What are the initial steps the CSIRT team will take to contain the damage caused by the incident?
	
	\item What evidence might the team need to collect and preserve from the incident?
	
	\item Outside of your organisation, who may need to be contacted as part of the incident response?  What do they need to know?  Who should manage that communication?
	
	\item What final steps need to be taken to fully recover your orgainisations systems and restore normal operations?
	
	   
\end{enumerate}

\newpage 

\section{Incident}
Prepare a response plan for the following incident:

An attacker has apparently exploited a persistent cross site scripting vulnerability in your organisation's web site and used it to distribute malware to visitors' computers.

\section{Submission instructions}
Prepare a document outlining your response plan and submit a hard copy at the beginning of Thursday's class.  This document is intended to be an initial draft of a response plan that will be refined and developed further.  You may need to do some research about the nature of the security incident to prepare a plan.

To get full marks for this lab you need to address all questions above in a well written (with correct spelling and grammar, of course) and well formatted document. Although this is not a research paper, you should note and cite any references you use, since they may be valuable when working on a finished plan. Although there are no specific length requirements of limits, you should expect to submit about two A4 pages.
 
\end{document}
