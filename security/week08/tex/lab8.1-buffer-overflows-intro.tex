\documentclass{article}
\usepackage{enumerate}
\usepackage{verbatim}
\usepackage[parfill]{parskip}
\usepackage[margin = 2.5cm]{geometry}

\usepackage[T1]{fontenc}


\begin{document}

\title{ Lab 8.1: Introduction to Buffer Overflows\\ IN618 Security}
\maketitle

\section*{Introduction}
The vulnerabilities we have explored so far in this paper are fairly ``modern'' ones,
but in today's lab we will examine a ``classic'' vulnerability, the buffer overflow. 
Many of the most critical security holes in computer systems have come from buffer
overflow vulnerabilities. In fact, the first major Internet worm, the 1988 \emph{Morris Worm},
exploited a buffer overflow.

Another difference between a buffer overflow vulnerability and the vulnerabilities we have
looked at so far this semester is that a buffer overflow requires a higher degree of technical
sophistication to expoit it.

\section{A vulnerable example}
Using a text editor, write the C program below. Save your file with the name \texttt{overflow.c}.

\emph{file: overflow.c}\footnote{Source: http://www.thegeekstuff.com/2013/06/buffer-overflow/} 
\begin{verbatim}

#include <stdio.h>
#include <string.h>

int main(void)
{
    char buff[15];
    int pass = 0;
    printf("\nEnter the password : ");
    gets(buff);
    if(strcmp(buff, "in618"))
    {
            printf ("\nWrong Password \n");
    }
    else
    {
        printf ("\nCorrect Password \n");
        pass = 1;
    }

    if(pass)
    {
        /* Now Give root or admin rights to user*/
        printf ("\nRoot privileges given to the user \n");
    }

    return 0;
}

\end{verbatim}

\section{Compile and run the program}
We will compile and run our program on the \texttt{sec-student.sqrawler.com} server.  Use  
WinSCP to upload your source code file to your home directory on the server. Then, use 
PuTTY to get a shell session on your server.  Compile your program with the command

\texttt{gcc -fno-stack-protector -o overflow overflow.c}

You will get a warning message, but your program should compile correctly provided that you
typed it in without errors.

This will produce an executable file named \texttt{overflow}.  Run the program with the 
command

\texttt{./overflow}

Try it first by entering the correct password and see what happens.  Next try an incorrect password,
using five \texttt{a} characters and see what happens.

Now let's try something more interesting.  Run the program and use 32 \texttt{a} characters.  What 
happens this time?  Do you know why?

What happens when you try 60 \texttt{a}s?

\section{A little debugging}
Whether or not the program gives us ``Root privileges'' depends on the variable \texttt{pass}.
Let's see what is happening to it.  In your source code file, right after the line the reads
\texttt{gets(buff)}, add the line

\begin{verbatim}
printf("pass:  %d\n", pass);   
\end{verbatim}

Then recompile your program and re-run your tests.  What is happening with \texttt{pass}?
Do you know why?
\end{document}
