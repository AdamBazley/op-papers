% Beamer slide template prepared by Tom Clark <tom.clark@op.ac.nz>
% Otago Polytechnic
% Dec 2012

\documentclass[10pt]{beamer}
\usetheme{CambridgeUS}
\usepackage{graphicx}
\usepackage{fancyvrb}


\title{Key Management and PKI}

\author[IN618]{Introduction to IT Security}
\institute[Otago Polytechnic]{
  Otago Polytechnic \\
  Dunedin, New Zealand \\
}
\date{}
\begin{document}

%----------- titlepage ----------------------------------------------%
\begin{frame}[plain]
  \titlepage
\end{frame}


%----------- slide --------------------------------------------------%
\begin{frame}
  \frametitle{From last week: public key cryptography}

 \begin{itemize}
  \item Recall that we have two kinds of keys:
    \begin{enumerate}
      \item Public keys
      \item Private keys
    \end{enumerate}
  \item Data encrypted with one key are decrypted with the other.
  \item If I encyrpt a message with my private key, you \textbf{know} it came from me.
  \item If you encrypt a message with my public key, you \textbf{know} that only I can read it.
  \end{itemize}
\end{frame}


%----------- slide --------------------------------------------------%
\begin{frame}
  \frametitle{How can we \emph{know} these things?}

 \begin{itemize}
  \item Last week I compromised someone's public key during our lab.
  \item Public key crypto depends on your ability to \textbf{trust} my public key.
  \item We need some mechanism to build that trust.
  \end{itemize}
\end{frame}


%----------- slide --------------------------------------------------%
\begin{frame}
  \frametitle{DIY approach:  key signing parties}

 \begin{itemize}
  \item We can all meet, face-to-face, confirm our identities, and exchange public keys.
  \item You go home afterward and \emph{sign} the keys you received - meaning you verify (to yourself) their authenticity.
  \end{itemize}
\end{frame}


%----------- slide --------------------------------------------------%
\begin{frame}
  \frametitle{Web of trust}

 \begin{itemize}
  \item Suppose that you and I exchange keys, so now you trust my key.
  \item Suppose that I also have and trust Darrell's key, but you do not.
  \item I can sign Darrell's key with my own and pass it on to you.
  \item Since you trust my key, and I've told you that I trust Darrell's key, you can now trust Darrell's key.
  \end{itemize}
\end{frame}


%----------- slide --------------------------------------------------%
\begin{frame}
  \frametitle{How do we scale this up?}
  
  This is great, but it won't scale.  Amazon.com can't rely on everybody getting a personal introduction.
\end{frame}



%----------- slide --------------------------------------------------%
\begin{frame}
  \frametitle{Certificate Authorities (CAs)}

 \begin{itemize}
  \item A certificate authority is an organisation that acts as a \emph{trusted third party} that issues cryptographically signed \emph{digital certificates}.
  \item Examples include Symantec, Comodo SSL, and GoDaddy.
  \item If you need a cetrtificate, for example to run a secure web site, you verify yourself to a CA, pay a fee, and get a certificate.
  \item CA's also maintain \emph{Certificate Revocation Lists} - lists of certificates that are no longer valid.
  \end{itemize}
\end{frame}



%----------- slide --------------------------------------------------%
\begin{frame}[fragile]
  \frametitle{X.509}

  Digital certificates generally follow the X.509 standard.

  Example:  \url{http://en.wikipedia.org/wiki/X.509#Sample_X.509_certificates}
\end{frame}

%----------- slide --------------------------------------------------%
\begin{frame}
  \frametitle{Questuions?}
  
   We will see some client details of digital certificates and PKI in the lab.

\end{frame}
\end{document}
