% Beamer slide template prepared by Tom Clark <tom.clark@op.ac.nz>
% Otago Polytechnic
% Dec 2012

\documentclass[10pt]{beamer}
\usetheme{Dunedin}
\usepackage{graphicx}
\usepackage{fancyvrb}

\newcommand\codeHighlight[1]{\textcolor[rgb]{1,0,0}{\textbf{#1}}}

\title{Incident Response}

\author[IN618]{Security}
\institute[Otago Polytechnic]{
  Otago Polytechnic \\
  Dunedin, New Zealand \\
}
\date{}
\begin{document}

%----------- titlepage ----------------------------------------------%
\begin{frame}[plain]
  \titlepage
\end{frame}

\begin{frame}
	\frametitle{Problem scenario}
	
	Suppose that a significant and apparently fast spreading malware
	outbreak has affected the Polytech campus.  You are an IT staffer
	who has been called in to respond to the problem.  What do you do?
	
	\vspace{5mm}
	Organise into 2 - 3 person groups to discuss the actions you will
	take to deal with this problem.
\end{frame}





\begin{frame}
	\frametitle{Here's the thing...}
	
	\begin{itemize}
		\item You're going to face security incidents during your careers.
		\item When you do, you're going to have to act quickly and decisively.
		\item At the same time you're going to be under stress, and most people
		      don't perform that well under stress.
		\item You need to make a plan \emph{before} these things happen.
	\end{itemize}
\end{frame}

\begin{frame}
	\frametitle{Form a team}
	
	To deal with security incidents, organisations need to form a Computer Security
	Incident Response Team (CSIRT). These teams generally include the following roles:
	
	\begin{itemize}
		\item Team Leader
		\item Incident Lead
		\item IT Staff Contact
		\item Legal Representative
		\item Public Relations Officer
		\item Management Representative
	\end{itemize}
\end{frame}

\begin{frame}
    \frametitle{Team Priorities}
     
     To form a response plan we have to identify a list of priorities.  Different
     organisations and different incident types have different priorities, but here's
     a starting point:
     
     \begin{enumerate}
     	\item Protect the safety of people.
     	\item Protect sensitive data.
     	\item Protect other data.
     	\item Protect systems from damage.
     	\item Minimise disruption to business.
     \end{enumerate}
 \end{frame}
     
\begin{frame}
	\frametitle{Response process}
	
	1. Relevant staff throughout the organisation need
	to know how to identify and report possible security 
	incidents.
	
\end{frame}

\begin{frame}
	\frametitle{Response process}
	
	2. The CSIRT team performs an initial assessment of the problem and 
	communicates its findings.  Assessment criteria need to be defined as
	part of the plan.

\end{frame}

\begin{frame}
	\frametitle{Response process}
	
	3. Contain damage and manage further risk.
\end{frame}

\begin{frame}
	\frametitle{Response process}
	
	4. Gather information and protect evidence.
\end{frame}

\begin{frame}
	\frametitle{Response process}
	
	5. Notify external parties as necessary.
\end{frame}

\begin{frame}
	\frametitle{Response process}
	
	6. Recover systems.
\end{frame}

\begin{frame}
	\frametitle{Response process}
	
	7. Perform a postmortem analysis.
\end{frame}

\begin{frame}
	\frametitle{Response process}
	
	After operations are restored, use what was learned to 
	update and improve 
	\begin{itemize}
		\item systems;
		\item policies and procedures;
		\item security response plans.
	\end{itemize}
\end{frame}

\begin{frame}
	\frametitle{CSIRT Resources}
	
	The CSIRT team will need resources for recovery, analysis,
	and communication.  Since the organisation's primary ICT resources may be
	compromised, you need to prepare seperate resources.
\end{frame}
	
	\begin{frame}
		\frametitle{More information}
		
		\begin{itemize}
			\item http://www.ncsc.govt.nz/assets/NCSC-Documents/ \\
			New-Zealand-Security-Incident-Management-Guide-for- \\
			Computer-Security-Incident-Response-Teams-CSIRTs.pdf 
			\item https://technet.microsoft.com/en-us/ \\ library/cc700825.aspx
		\end{itemize}
	\end{frame}
	
\end{document}

