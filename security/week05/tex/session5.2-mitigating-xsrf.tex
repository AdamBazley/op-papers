\documentclass{article}
\usepackage{graphicx}
\usepackage{wrapfig}
\usepackage{inconsolata}
\usepackage{enumerate}
\usepackage{hyperref}
\usepackage{verbatim}
\usepackage[parfill]{parskip}
\usepackage[margin = 2.5cm]{geometry}

\usepackage[T1]{fontenc}


\begin{document}

\title{Cross Site Request Forgery: Lab Exercise\\ IN618 Security}
\date{}
\maketitle

\section*{Introduction}
Last time we saw how an XSRF vulnerability can be exploited.  In this exercise we will
take the code for the vulnerable application and modify it to remove the XSRF vulerability
by adding an extra token to our forms. The key idea is that we will add and track a bit more
state as the user traverses the application.

All work will be done the the file \texttt{public\_html/xsrf/home.php}.  This file
and all others needed to run and test your work have already been placed in the \texttt{xsrf}
directory inside your own \texttt{public\_html} directory.

\section{Tasks}
First we need to be able to track user sessions and store information about them on the server.
Fortunately, we do not have to implement this ourselves, since PHP includes this as a core feature.
Just add the line

\texttt{ session\_start(); } 

at the very beginning of the file. This will cause the server to send a session cookie
to the user's browser and to store session information on the server.

Next, we need to add a special session token to the web forms we produce on this page. Add the
code below after line 31 (assuming you've added only the one line above so far).

\begin{verbatim}
 $xsrf_token = md5($token . rand() . microtime());
 $_SESSION['saved_xsrf_token']=$xsrf_token;
\end{verbatim}

The first line creates a reasonably unpredictable token that we can add to our forms.  We
take the user's login token and concatenate it with a random number and the server's system
time in microseconds.  We then MD5 hash that value.  On the second line we save that value in
the user's session.  We can retrieve that value when the page loads again later.

Now we need to add the token value to the HTML forms we write.  There are two such forms
on this page.  To each one, add the following:

\begin{verbatim}
 <input type="hidden" name="xsrf_token" value="<?php echo($xsrf_token) ?>" />
\end{verbatim}

So now, in a way that is largely transparent to the users, we include this token value
with their form submission to prove that they submitted this form properly.  If an 
attacker could intercept this value along with the user's session cookie and login cookie
she could spoof the form submission, but only for a short time since the XSRF token changes
every time the user loads the page.

\newpage 

Now we just need to verify the XSRF token when we process the form submissions.  We do this at about 
line 26. Find that part of the code, where we look at the \texttt{\$action} value. Wrap that entire
\texttt{if-elsif} block with an outer if test like this:

\begin{verbatim}
   if($_SESSION['saved_xsrf_token'] == $_REQUEST['xsrf_token']) {

     # form handling stuff that's already there

   }
\end{verbatim}

We compare the token submitted in the request (\texttt{\$\_REQUEST['xsrf\_token']})
with the one stored in the user session  (\texttt{\$\_SESSION['saved\_xsrf\_token']}) to 
see that they match.

This is all the code we need to add.  Test your modifications to see that they work.  Be sure
to view the HTML source of your forms to see that the new XSRF token is included.  Both forms
submit GET requests so that you can see the form fields that are submitted. Check to see that
you can copy those urls and try submitting them directly, but they don't work.  Do you know why 
that is?

\section{Submission}
In addtion to testing your code on the server, submit a copy of your code to your SVN repository by
Thursday, 26 March.

\section{Resources}
The files you need to start this assignment have been copied to your \texttt{public\_html} directory
on the \texttt{sec-student} server.

\section{Marking}

There are 10 marks for this exercise:

\begin{itemize}
	\item Modifying and testing your code on the \texttt{sec-student} server: 5 marks;
	\item Committing correct and well-formatted code to SVN : 5 marks;
\end{itemize}

\end{document}
