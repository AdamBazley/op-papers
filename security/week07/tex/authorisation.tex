% Beamer slide template prepared by Tom Clark <tom.clark@op.ac.nz>
% Otago Polytechnic
% Dec 2012

\documentclass[10pt]{beamer}
\usetheme{CambridgeUS}
\usepackage{graphicx}
\usepackage{fancyvrb}


\title{Authorisation}

\author[IN618]{Introduction to IT Security}
\institute[Otago Polytechnic]{
  Otago Polytechnic \\
  Dunedin, New Zealand \\
}
\date{}
\begin{document}

%----------- titlepage ----------------------------------------------%
\begin{frame}[plain]
  \titlepage
\end{frame}


%----------- slide --------------------------------------------------%
\begin{frame}
  \frametitle{From last time}

 AAA

 \begin{itemize}
	 \item Authentication: who you are 
	 \item Authorisation: what you can do
	 \item Accounting: tracking what you have done
 \end{itemize}
\end{frame}



%----------- slide --------------------------------------------------%
\begin{frame}
	\frametitle{Authorisation: Strategic concerns}

 A lot of the security literature seems to focus on access to information, i.e., what you can see.
 This is the wrong way to think about authorisation.  Instead, think about controlling access to 
 \emph{actions}. (Think verbs, not nouns.)

 Example:  My access to the student database
 \begin{itemize}
  \item Action:  create a new student record - not authorised		 
  \item Action:  read your student record - authorsised
  \item Action:  update  your student record - not authorsised
  \item Action:  delete your student record - not authorsised
  \end{itemize}
\end{frame}


%----------- slide --------------------------------------------------%
\begin{frame}
  \frametitle{Authorisation: Strategic concerns}

 Least privilege approach
 \begin{itemize}
  \item The default answer to ``Can I do this?'' should be ``No''.
  \item Selectively choose to say ``Yes'' only when the requestor 
	  \begin{itemize}
		  \item is trusted
		  \item needs access
	  \end{itemize}

					  
  \end{itemize}
\end{frame}



%----------- slide --------------------------------------------------%
\begin{frame}
  \frametitle{Authorisation: Policy}

 The theory is that \emph{management}, not IT, should set authorisation policies.  However, in 
 practice
 \begin{itemize}
  \item Managers won't set policy because they doesn't realise that they need to, and
  \item will get it wrong if they try.
  \end{itemize}

  So as IT people we should be prepared to guide management through the policy making process.
  This doesn't mean that IT should take over.  Management should decide, with help from IT.
\end{frame}



%----------- slide --------------------------------------------------%
\begin{frame}
  \frametitle{Authorisation Types: DAC}

  Discretionary Access Control

  \begin{itemize}
  \item A user has control of certain resources.
  \item The user may choose to grant some control to other users or groups.
  \item The grants are generally tracked with an \emph{Access Control List} (ACL).
  \item DAC is the access control method used by nearly all operating systems.
  \end{itemize}
\end{frame}



%----------- slide --------------------------------------------------%
\begin{frame}[fragile]
  \frametitle{}

 \begin{verbatim}
 tclark@biblios:~$ ls -l /tmp
 total 16
 -rw-r--r-- 1 tclark staff     0 Sep  5 11:06 examplefile
 drwx------ 2 tclark tclark 4096 Sep  5 08:07 i3-tclark.KUjqAK
 drwxr-xr-x 3 root   root   4096 Sep  5 08:05 passenger.1.0.2488
 drwx------ 2 tclark tclark 4096 Sep  5 10:17 pulse-FdFIn2LtSoNz
 drwx------ 2 tclark tclark 4096 Sep  5 08:07 ssh-8C0VROBLuELE

  \end{verbatim}
\end{frame}

%----------- slide --------------------------------------------------%
\begin{frame}
  \frametitle{Authorisation Types: MAC}

  Mandatory Access Control

  \begin{itemize}
  \item All resources are tagged with security attributes.
  \item All users are also tagged with secutiry attributes.
  \item Whenever a user attempts to access a resource, the secutity attributes of the 
	  user and resource are checked to see if the access is authorised.
  \item Attributes and policies are manged by a security administrator. Users
	  can not override or changes security attributes and policies.
  \item Examples include SELinux, FreeBSD.

  \end{itemize}
\end{frame}



%----------- slide --------------------------------------------------%
\begin{frame}
  \frametitle{Authorisation Types: RBAC}

  Role-Bases Access Control

 \begin{itemize}
	 \item Users are assigned to a number of \emph{roles}, perhaps by job function.
	 \item Access to resources is permitted for users who have the allowed role.
	 \item Example: Practically every web content mangement system.
  \end{itemize}
\end{frame}

%----------- slide --------------------------------------------------%
\begin{frame}
	\frametitle{Authorisation Types: RBAC\footnote{The same intialism twice.  Wow, that sucks.}}

  Rule-Bases Access Control

 \begin{itemize}
	 \item The access policy is built around a number of rules that govern access. 
	 \item Examples:
		 \begin{itemize}
			 \item Users may be able to access a resource during certain hours of the day.
			 \item Users may be able to access a resource from specified network locations.
		 \end{itemize}
  \end{itemize}
\end{frame}


%----------- slide --------------------------------------------------%
\begin{frame}
  \frametitle{Today's Lab}

  Design an R(ole)BAC scheme for a new College of EAD Content Management System.
 \begin{itemize}
  \item The system will allow staff to create, edit, pubilsh, unpublish, and remove web pages.
  \item The system will have four sections:  Whole College, BIT, BAM, and Certificate Programs.
  \item Ordinary staff will be able to create and edit page drafts in their area.
  \item Team leaders will be able to publish drafts so that they appear on the web site and unpublish theem.
  \item The Head of College, Leslie, and her designates shall have global rights to do anything.
  \end{itemize}

Your task is to identify a set of roles for this system.
 \begin{itemize}
	 \item List the roles.
	 \item Identify which staff may be in which roles.  (N.B.: A person may have multiple roles)
	 \item Make a (preliminary) list of the tasks associated with each role.
  \end{itemize}

\end{frame}
\end{document}

