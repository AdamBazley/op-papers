\documentclass{article}
\usepackage{graphicx}
\usepackage{wrapfig}
\usepackage{inconsolata}
\usepackage{enumerate}
\usepackage{hyperref}
\usepackage{verbatim}
\usepackage[parfill]{parskip}
\usepackage[margin = 2.5cm]{geometry}

\usepackage[T1]{fontenc}


\begin{document}

\title{Cross Site Scripting: Lab Exercise\\ IN618 Security}
\date{}
\maketitle

\section*{Introduction}
The marketing manager got a copy of \emph{Head First PHP \& MySQL} last week
and now he thinks he's a programmer.  His first project is a product review
page that allows visitors to your company's web site to submit reviews.

Appeals to higher-ups for sanity have failed, so now your team must
add this feature to the web site.  You have been given the task of 
reviewing the code and correcting security mistakes before the pages
go live.

\section{Tasks}
The initial code is available in the IN618 directory on the I: drive in a file 
named \texttt{xss-assignment.zip}. Get of copy of the code and assess its
vulnerabilities to XSS exploits.  Modify it so that

\begin{enumerate}
	\item Vulnerabilities to XSS exploits are removed;
	\item Input data is sanitised appropriately.
\end{enumerate}

You can reasonably determine data validity criteria by inspecting the code.

\section{Submission}
Upload your completed code to your \texttt{public\_html} directory on 
\texttt{sec-student.sqrawler.com}.  Your files must be in a subdiectory
named \texttt{xss-assignment}. Leave the filenames as they are in the 
original source files you received.  Failure to follow these instructions
will interfere with testing your work and will result in a loss of marks.

Your code must be uploaded and ready for testing by 1:00 PM on Monday,
16 March.

\section{Resources}
You should already have an account on the web server at \texttt{sec-student.sqrawler.com}.
Upload your code there for testing and final submission.

\newpage

\section{Marking}

There are 10 marks for this exercise:

\begin{itemize}
	\item Sucessfully resisting XSS attacks performed by the lecturer: 5 marks;
	\item Correctly handling and sanitising user input data: 4 marks;
	\item Well formatted and readable code: 1 mark.
\end{itemize}

\end{document}
