% Beamer slide template prepared by Tom Clark <tom.clark@op.ac.nz>
% Otago Polytechnic
% Dec 2012

\documentclass[10pt]{beamer}
\usetheme{Dunedin}
\usepackage{graphicx}
\usepackage{fancyvrb}

\newcommand\codeHighlight[1]{\textcolor[rgb]{1,0,0}{\textbf{#1}}}

\title{Algorithm Analysis}

\author[IN711]{Algorithms and Data Structures}
\institute[Otago Polytechnic]{
  Otago Polytechnic \\
  Dunedin, New Zealand \\
}
\date{}
\begin{document}

%----------- titlepage ----------------------------------------------%
\begin{frame}[plain]
  \titlepage
\end{frame}


%----------- slide --------------------------------------------------%
\begin{frame}
  \frametitle{What is ADS about?}

  \begin{itemize}
    \item This class is about writing quality code.
    \item In OOSD we looked at larger scale architectural issues.
    \item In ADS we will zoom in and look at the details of efficient 
          code.
  \end{itemize}
\end{frame}



%----------- slide --------------------------------------------------%
\begin{frame}
	\frametitle{Algorithms and Data Structures}
	
	\begin{itemize}
		\item Algorithms are step-by-step instructions to perform a task.
		\item Data structures are used to organise collections of data.
		\item We will see that a good choice of data structure can help us use 
		      a good algorithm.
			
	\end{itemize}
\end{frame}


%----------- slide --------------------------------------------------%
\begin{frame}
	\frametitle{Evaluating Efficiency }
	
	\begin{itemize}
		\item For a given algorithm, we count the number of ``primitive operations'' performed.
		\item In general, this number is a function of the number of inputs into the algorithm.
		\item An efficient algorithm is one for which the number of operations doen't increase too quickly.
		\item We generally compare the function to well-known mathematical functions. 
	\end{itemize}
\end{frame}
	
%----------- slide --------------------------------------------------%
\begin{frame}
	\frametitle{Key Functions}
	\begin{enumerate}
		\item $f(x) = c$
		\item $f(x) = \log_{2}(x)$
		\item $f(x) = x$
		\item $f(x) = x\log_{2}(x)$
		\item $f(x) = x^{2}$
		\item $f(x) = x^{k}, k > 2$
		\item $f(x) = 2^{x}$
	\end{enumerate}
\end{frame}

\begin{frame}
	\frametitle{Big-O, Informally}
	
	Suppose we analyse the number of primitive operations performed by an 
	algorithm on a list of $n$ items and find that it is characterised by the 
	function 
	
	$f(n) = 4n^{2} + 3n + 6$
	
	In this case we would say that the algorithm is Big-O of $x^{2}$, because in Big-O
	analysis we ignore coefficients and lower order terms.
\end{frame}

\begin{frame}
	\frametitle{Big-O, Formally}
	
	Given function $f(n):\Bbb Z^{+}\rightarrow\Bbb R, g(n):\Bbb Z^{+}\rightarrow\Bbb R$, we say that 
	$f(n)$ is $O(g(n))$ if there is a real constant $c>0$ and an integer $n_{0}$ such that
	
	\vspace{5mm}
	
	$f(n) \leqq cg(n)$, for $n \geqq n_{0}$
	
\end{frame}

\begin{frame}
	\frametitle{Exercises}
	
	R-3.22 through R-3.27
	
	C-3.36
	
	C-3.42
	
	C-3.45
	
	\vspace{5mm}
	Submit your answers to these and be ready to discuss your solutions next time.
	
\end{frame}
\end{document}
