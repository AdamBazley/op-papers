% Beamer slide template prepared by Tom Clark <tom.clark@op.ac.nz>
% Otago Polytechnic
% Dec 2012

\documentclass[10pt]{beamer}
\usetheme{Dunedin}
\usepackage{graphicx}
\usepackage{fancyvrb}

\newcommand\codeHighlight[1]{\textcolor[rgb]{1,0,0}{\textbf{#1}}}

\title{Exception Handling}

\author[IN710]{Object Oriented System Design}
\institute[Otago Polytechnic]{
  Otago Polytechnic \\
  Dunedin, New Zealand \\
}
\date{}
\begin{document}

%----------- titlepage ----------------------------------------------%
\begin{frame}[plain]
  \titlepage
\end{frame}

%----------- slide --------------------------------------------------%
\begin{frame}
  \frametitle{Introduction}

 \begin{itemize}
	 \item Things can go wrong in our programs.
	 \item Consider ex1.py in the example code on GitHub.
	 \item You can anticpate ways to break it before you even run it.
	 \item Run it anyway.  Break it and see what happens.
  \end{itemize}
\end{frame}

\begin{frame}
	\frametitle{Exceptions}
	\begin{itemize}
		\item When something goes wrong in a Python program
			it \emph{raises an exception}.
		\item If the exception isn't handled the program terminates.
		\item Sometimes this is unavoidable or even desirable,
			but we don't want this to happen all the time.
	\end{itemize}
\end{frame}

\begin{frame}
	\frametitle{Guarding against exceptions}
	\begin{itemize}
		\item When we're programming we can usually identify
			places in our code where exceptions are likely 
			to occur.
		\item Consider ex2.py.  We use an \texttt{if-elif-else} 
			structure to guard against likely problems.
		        Run it and see if you can get it to crash.\footnote{I can.}
		\item This version is far more robust, but there is a cost. 
			There are only three lines of code that do what we
			really want mixed in with five lines of code that
			protect those three.
	\end{itemize}
\end{frame}


\begin{frame}
	\frametitle{Exception handling}
	\begin{itemize}
		\item Python, like many languages, allows you to
			handle excetions in your code and avoid
			crashing.
		\item Consider ex3.py
		\item We still have more error handling code than
			``normal'' code, but we have seperated the two
			so that the main code path is clear.
	\end{itemize}
\end{frame}

\begin{frame}
	\frametitle{Exception handling - mixed approach}
	\begin{itemize}
		\item It's best to use exception handling for 
			cases that are truly \emph{exceptional}.
		\item Consider ex4.py. We use a \texttt{while} to 
			handle the more common user error.  We use
			\texttt{try/except} to handle what we think is a less
			common error.

	\end{itemize}
\end{frame}

\begin{frame}
	\frametitle{Practical exercise}
	\begin{enumerate}
		\item Write a function called \texttt{oops} that 
			explicitly raises an \texttt{IndexError}
			exception when called. Then write another 
			function that calls oops inside a \texttt{try/except}
			statement to catch the error. What happens if you 
			change oops to raise \texttt{KeyError} instead of 
			\texttt{IndexError}?

		\item Change the \texttt{oops} function you just wrote to 
			raise an exception you define yourself, 
			called \texttt{MyError}, and pass an extra data item along 
			with the exception. Then, extend the \texttt{try} 
			statement in the catcher function to catch 
			this exception and its data in addition to 
			i\texttt{IndexError}, and print the extra data item.
		\item Modify your \texttt{MyError} exception class so that it
			inherits from \texttt{IndexError} instead of
			\texttt{Exception}.  What happens when it is thrown?
	\end{enumerate}

	Save your code examples in a directory called \texttt{exceptions} and
	push it to your GitHub repository.
\end{frame}


\end{document}
