% Beamer slide template prepared by Tom Clark <tom.clark@op.ac.nz>
% Otago Polytechnic
% Dec 2012

\documentclass[10pt]{beamer}
\usetheme{Dunedin}
\usepackage{graphicx}
\usepackage{fancyvrb}

\newcommand\codeHighlight[1]{\textcolor[rgb]{1,0,0}{\textbf{#1}}}

\title{Core OOSD: SOLID}

\author[IN710]{Object Oriented System Design}
\institute[Otago Polytechnic]{
  Otago Polytechnic \\
  Dunedin, New Zealand \\
}
\date{}
\begin{document}

%----------- titlepage ----------------------------------------------%
\begin{frame}[plain]
  \titlepage
\end{frame}

%----------- slide --------------------------------------------------%
\begin{frame}
  \frametitle{Background}

 \begin{itemize}
  \item OO Programming emerged in the 1980's qnd 1990's.
  \item By the early 2000's programmers were building larger programs and started to recognise 
	  design problems.
  \item Robert Martin (``Uncle Bob'') identified five key principles that have become known
	  by the acronym SOLID.
  \end{itemize}
\end{frame}


\begin{frame}
	\frametitle{Guidlelines}

	\begin{itemize}
		\item It's a bit of a clich\'{e} that almost any question in an advanced
			programming class can be answered, ``It depends.''
		\item How closely should you follow the guidlelines we will discuss
			this morning?  It depends.
		\item You should try to follow them. When you do find yourself breaking them,
			it should be because you've made a deliberate choice to do so.
	\end{itemize}
\end{frame}

\begin{frame}
	\frametitle{SOLID}
	\begin{itemize}
		\item \textbf{S}ingle responsibility principle
		\item \textbf{O}pen/closed principle
		\item \textbf{L}iskov substitution principle
		\item \textbf{I}nterface substitution principle
		\item \textbf{D}ependency inversion principle
	\end{itemize}
\end{frame}

\begin{frame}
	\frametitle{Single responsibility principle}
	\begin{itemize}
		\item A class should do only one thing.
		\item You should be able describe a class' purpose with
			one \emph{concise} sentence.
		\item When specifications change, a class should have only
			one reason to change.
	\end{itemize}
\end{frame}

\begin{frame}
	\frametitle{Example: Card class}
	\begin{itemize}
		\item What is the job of the playing card?
		\item Should it report its numeric score value for blackjack?
		\item This would mean that the cards would need to
			know about blackjack rules.
	\end{itemize}
\end{frame}

\begin{frame}
	\frametitle{Open/closed principle}
	\begin{itemize}
		\item Classes should be \emph{open} to extension, but \emph{closed} 
			to modification.
		\item This means that consumers of a class can rely on its
			methods remaining available, i.e., changes to the class
			won't hurt current uses of the class.
		\item We can extend a class so that it can do more, but we never
			take away or modify exisiting functionality.
	\end{itemize}
\end{frame}

\begin{frame}
	\frametitle{Example: Deck class}
	\begin{itemize}
		\item I wrote my \texttt{Deck} class to use one 52 card deck.
		\item Casinos usually use multiple decks to make card counting harder.
		\item But if I \emph{modify} the deck class to use more decks,
			I may break exisiting uses of the class.
		\item However, I could \emph{extend} the class, for example by 
			supplying an alternate constructor, without breaking 
			preexisiting uses.
	\end{itemize}
\end{frame}
\begin{frame}
	\frametitle{Liskov substitution principle}
	\begin{itemize}
            \item Objects in a program should be replaceable with instances of their subtypes without altering the correctness of that program.
	\end{itemize}
\end{frame}
\begin{frame}
	\frametitle{Interface segregation principle}
	\begin{itemize}
            \item Many client-specific interfaces are better than one general-purpose interface.
	    \item Clients should not be forced to depend upon methods they do not use.
	\end{itemize}
\end{frame}
\begin{frame}
	\frametitle{Dependency inversion principle}
	\begin{itemize}
            \item High level modules should not depend on low-level modules.
	    \item Both should depend upon abstractions.
	\end{itemize}
\end{frame}

\end{document}
