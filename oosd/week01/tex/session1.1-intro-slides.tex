% Beamer slide template prepared by Tom Clark <tom.clark@op.ac.nz>
% Otago Polytechnic
% Dec 2012

\documentclass[10pt]{beamer}
\usetheme{Dunedin}
\usepackage{graphicx}
\usepackage{fancyvrb}

\newcommand\codeHighlight[1]{\textcolor[rgb]{1,0,0}{\textbf{#1}}}

\title{Introduction}

\author[IN710]{Object Oriented System Design}
\institute[Otago Polytechnic]{
  Otago Polytechnic \\
  Dunedin, New Zealand \\
}
\date{}
\begin{document}

%----------- titlepage ----------------------------------------------%
\begin{frame}[plain]
  \titlepage
\end{frame}

\section{Course information}
%----------- slide --------------------------------------------------%
\begin{frame}
  \frametitle{Learning goals}

 \begin{itemize}
  \item Develop more advanced programming skills
  \item Gain understanding of and experience with deeper
	  OO architectural principles
  \item Learn how to produce professional quality OO code
  \end{itemize}
\end{frame}

\begin{frame}
	\frametitle{Course content}
	\begin{itemize}
		\item Key themes
			\begin{itemize}
				\item Design patterns
				\item Advanced programming topics
				\item Unit testing
			\end{itemize}
		\item Recommended books
			\begin{itemize}
				\item \emph{Design Patterns: Elements of
						Reusable Object-Oriented
					Software} Erich Gamma, Richard Helm
					Ralph Johnson, and John Vlissides
				\item \emph{Head First Design Patterns}
					Eric Freeman, Elisabeth Freeman
				\item \emph{Learning Python} 5th ed. 
					Mark Lutz
			\end{itemize}
		\item Required readings will be provided and are testable
	\end{itemize}
\end{frame}

\begin{frame}
	\frametitle{Assessments}
	\begin{itemize}
		\item Weekly labs (p/f): 25\%
		\item Project: 45\%
		\item Final exam: 30\%
	\end{itemize}
\end{frame}



\section{OO Design}
\begin{frame}
	\frametitle{Real programming}
	\begin{itemize}
		\item You're on the threshhold of writing genuinely 
			\emph{interesting} programs.
		\item These programs get so big that you simply
			cannot keep track of how everything works
			at any given time.
		\item You will also work in groups where no one person
			knows how everything works.
		\item We need reliable, \emph{widely recognised} design
			patterns 
	\end{itemize}
\end{frame}

\begin{frame}
	\frametitle{The value of good design}
	\begin{itemize}
		\item The classic literature on OO design and 
			patterns talks about ``code reuse'' and 
			the need to decouple functionality from 
			implementation.
		\item This a largely predicated on the notion that 
			``programming is hard''.
		\item The problem with this view is that while bending 
			over backwards to avoid locking us into a 
			concrete implementation of a type, we write 
			unreadable code\footnote{I'm looking at you,
			abstract factory pattern.}.
	\end{itemize}
\end{frame}
\begin{frame}
	\frametitle{The value of good design}
	\begin{itemize}
		\item The thing is, writing code is not the hard part.
		\item \emph{Reading} code is hard, and it turns out
			to be the more important part. Nearly everything
			that we've taught you about coding so far is about
			writing it to be read by people.
		\item Done well, good OO design should lead to more readable
			code.
		\item When adhering to established design patterns doesn't 
			help readability, then it's time to chuck it over the 
			side.
	\end{itemize}
\end{frame}
\begin{frame}
	\frametitle{Readability and quality}

         More succinctly, readability may be the strongest indicator of
	 quality design and code.
 \end{frame}

 
\section{Core OO Principles}
\begin{frame}
	\frametitle{Five big ideas}
	\begin{enumerate}
		\item Encapsulation
		\item Inheritance
		\item Composition
		\item Association
		\item Polymorphism
	\end{enumerate}
\end{frame}
\begin{frame}
	\frametitle{Encapsulation}

	Encapsulation is the idea that an object exposes
	an \emph{interface} without exposing its \emph{implementation}.
\end{frame}
\begin{frame}
	\frametitle{Encapsulation}

	\begin{itemize}
		\item Suppose you want to represent a circle on a plane.
		\item You could store the radius and the coordinates
			of the centre.  Everything else (e.g., area,
			circumference) can be calculated on the fly.
		\item On the other hand, you could determine and save those values
			when the object is created.
		\item A user of your class would just call its \texttt{area()}
			function without regard to how it works.
	\end{itemize}
\end{frame}
\begin{frame}
	\frametitle{Inheritance}

	\begin{itemize}
		\item Inheritance captures an ``is-a'' relationship.
		\item For example, we could have a \texttt{Person} class that 
			encapsulates all of the behaviours common to people.
		\item Then we could have \texttt{Student} and \texttt{Staff} classes
			that inherit from \texttt{Person}. 
		\item We don't need to re-code the functionality that is implemented in \texttt{Person}
			within the child classes.
		\item The child classes \emph{extend} the parent class by adding behaviours specific
			to \texttt{Student} or \texttt{Staff}.
	\end{itemize}
\end{frame}
\begin{frame}
	\frametitle{Inheritance in practice}
	\begin{itemize}
		\item Although inheritance is often considered to be OO's
			killer feature, inpractice its utility is limited.
		\item I have found there are two patterns in which inheritance is 
			used effectively:
			\begin{enumerate}
				\item The parent class implements almost everything needed
					and child classes extend it in minor ways.
				\item The parent class implements a small core feature set,
					with almost all of the ``real'' work done by the child classes.
			\end{enumerate}
		\item TL;DR: use inheritance carefully and deliberately only when a true ``is-a'' 
			relationship exists - and sometimes not even then.
	\end{itemize}
\end{frame}
\begin{frame}
	\frametitle{Composition}
	\begin{itemize}
		\item Often we \emph{compose} objects out of smaller
			objects by including them as fields of the larger object.
		\item For example, in the past I mave modelled ski fields by composing them 
			from objects that respresent trails and chair lifts.
		\item In a Composition relationship, the smaller objects only exist in 
			the context of the larger aboject that is composed of them.
		\item A lot of situations that look like uses for inheritance turn out 
			to be better suited to composition.
	\end{itemize}
\end{frame}
\begin{frame}
	\frametitle{Association}
	\begin{itemize}
		\item Association is very similar to Composition.  
		\item In the code, often there is no difference.
		\item In an Association relationship, one object 
			knows about another object, but the second object is not a 
			part of the first.
		\item Example: Cars in a carpark.  The carpark ``knows'' about
			the cars parked within it, but the cars are not part of the 
			carpark.
		\item Notice that if we close the carpark, the cars still exist.
        \end{itemize}
\end{frame}

\begin{frame}
	\frametitle{Polymorphism}
	\begin{itemize}
		\item Polymorphism is usually thought of in connection with
			inheritance \emph{but it doesn't have to be.}
		\item The key idea is that we will call a method, but the 
			exact method called is not known until run time.
		\item Example:  Suppose we have a set of objects that model various 
			polygons of different types.  Each one implements an \texttt{area()}
			method. For a given polygon \texttt{p} we call \texttt{p.area()}, but
			the precise method called depends on the type of polygon.
	\end{itemize}
\end{frame}
\begin{frame}
	\frametitle{A quick aside: type vs. class}

	In many contexts, we use the terms \emph{type} and
	\texttt{class} as if they were the same thing. They
	are not.  Discuss.
\end{frame}

\section{Task One}
\begin{frame}

	Blackjack

\end{frame}
% tesla photo: https://www.flickr.com/photos/alabut/
\end{document}
