% Beamer slide template prepared by Tom Clark <tom.clark@op.ac.nz>
% Otago Polytechnic
% Dec 2012

\documentclass[10pt]{beamer}
\usetheme{Dunedin}
\usepackage{graphicx}
\usepackage{fancyvrb}

\newcommand\codeHighlight[1]{\textcolor[rgb]{1,0,0}{\textbf{#1}}}

\title{Introduction to Python}

\author[IN710]{Object Oriented System Design}
\institute[Otago Polytechnic]{
  Otago Polytechnic \\
  Dunedin, New Zealand \\
}
\date{}
\begin{document}

%----------- titlepage ----------------------------------------------%
\begin{frame}[plain]
  \titlepage
\end{frame}

\section{Introduction}
%----------- slide --------------------------------------------------%
\begin{frame}
  \frametitle{Python}

 \begin{itemize}
    \item Python is a widely used, object-oriented high level language.
    \item It is an expressive language that facilitates rapid development.
    \item The language syntax supports the Python community's stated goal
	    of producing readable code.
    \item Python is supported on many operating systems.
    \item Python is used at top-tier technology organisations like Google, NASA, Red Hat, 
	    and Rackspace.
  \end{itemize}
\end{frame}

\begin{frame}
	\frametitle{Python style}

	\begin{itemize}
		\item The Python language has a set of
			universal language-wide style guidelines.
		\item The style guidelines are published in the 
			\emph{Python Enhancement Proposal} (PEP) 8 at
			\url{https://www.python.org/dev/peps/pep-0008/}.
		\item In this class we will follow the PEP 8 style guidelines.
	\end{itemize}
\end{frame}

\begin{frame}
	\frametitle{The Python interpreter}
	\begin{itemize}
		\item Python is an interpreted language.
		\item You can intereact with the interpreter directly.
		\item This can be really useful when you are developing because you 
			can test little segments of code directly.
	\end{itemize}
\end{frame}

\section{Variables, types, and values}
\begin{frame}
	\frametitle{Variables}
	\begin{itemize}
		\item Python variable do not have types associated with them.
		\item They are not declared before use.
		\item To create a variable, just assign it a value.
		\item Python variable are references to objects.
	\end{itemize}
\end{frame}
\begin{frame}
	\frametitle{Types}
	\begin{itemize}
		\item Python values are strongly types and they are all objects.
		\item Numeric, string, and boolean types behave in the ways you would generally expect.
		\item There is no character type.  A character is just a short string.
		\item Python's collection types are powerful and flexible. Many problems that
			might require a class in other languages can be solved with a 
			basic collection in Python.
			\begin{itemize}
				\item List
				\item Tuple
				\item Dictionary
				\item Set, Frozenset
			\end{itemize}
	\end{itemize}
\end{frame}
\section{Control flow}
\begin{frame}
	\frametitle{White space}

	\begin{itemize}
		\item People who are new to Python are surprised by the 
			significant white space.
		\item The surprise wears off quickly.
		\item Statements are generally terminated by a line ending.
		\item Code blocks are denoted by indentation.
	\end{itemize}
\end{frame}
\begin{frame}[fragile]
	\frametitle{Conditionals}

	\begin{verbatim}
    if(a > 0):
        print("positive")
    elif(a < 0):
        print("negative")
    else:
        print("zero")

	\end{verbatim}
\end{frame}
\begin{frame}[fragile]
	\frametitle{For loops}

	\texttt{for} loops over anything that implements a sequence.
	\begin{verbatim}
    for i in xrange(10):
        print(i)

    for char in "spam":
        print(char)
	\end{verbatim}
\end{frame}
\begin{frame}[fragile]
	\frametitle{While loops}

	\begin{verbatim}
    while(stop != "y"):
        print("One more time.")
        stop = input("Go again? ")

	\end{verbatim}
\end{frame}
\section{Functions and classes}
\begin{frame}
	\frametitle{Functions}
	\begin{itemize}
		\item Define functions with the \texttt{def} keyword.
		\item Functions can return one or more values.
		\item Functions lacking an explicit return return the 
			special value \texttt{None}.
		\item Functions in Python are first-class.
	\end{itemize}
\end{frame}
\begin{frame}[fragile]
	\frametitle{Functions}

	\begin{verbatim}
    def times_two(x):
        return 2 * x
	\end{verbatim}
\end{frame}
\begin{frame}
	\frametitle{Classes}
	\begin{itemize}
		\item Define classes with the \texttt{class} keyword.
		\item Classes support inheritance, including multiple inheritance.
		\item Classes include class variables, instance variables, and
			methods.
	\end{itemize}
\end{frame}
\end{document}
