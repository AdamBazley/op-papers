% Beamer slide template prepared by Tom Clark <tom.clark@op.ac.nz>
% Otago Polytechnic
% Dec 2012

\documentclass[10pt]{beamer}
\usetheme{Dunedin}
\usepackage{graphicx}
\usepackage{fancyvrb}

\newcommand\codeHighlight[1]{\textcolor[rgb]{1,0,0}{\textbf{#1}}}

\title{Unit Testing}

\author[IN710]{Object Oriented System Design}
\institute[Otago Polytechnic]{
  Otago Polytechnic \\
  Dunedin, New Zealand \\
}
\date{}
\begin{document}

%----------- titlepage ----------------------------------------------%
\begin{frame}[plain]
  \titlepage
\end{frame}

%----------- slide --------------------------------------------------%
\begin{frame}
  \frametitle{We are bad coders and we should feel bad}

 \begin{itemize}
  \item Here in the BIT there are two things that we are gnerally bad at:
	  \begin{enumerate}
		  \item Documentation
		  \item Testing
	  \end{enumerate}
  \item I don't have any great ideas about documentation.  You just
	  have to suck it up.
  \item Testing is a different matter.  We must get better at testing. The
	  good news is that there are plenty of tools and methods to help us.
  \end{itemize}
\end{frame}

\begin{frame}
	\frametitle{Types of testing}
	\begin{itemize}
		\item Unit testing
		\item Integration testing
		\item Functional testing
		\item Performance testing
		\item Penetration testing
		\item Acceptance testing
	\end{itemize}
\end{frame}

\begin{frame}
	\frametitle{Things we do instead of testing}
	\begin{itemize}
		\item We write a lot of code in one go - maybe the whole assignment.
		\item When things don't work, we stare at the code and
			try to figure out why.
		\item We throw in some print statements.
		\item We use a debugger.
		\item We write throwaway examples to test things.
		\item In short we waste time.
	\end{itemize}
\end{frame}

\begin{frame}
	\frametitle{What we want to do.}
	\begin{itemize}
		\item We will write tests that cover all our code.
		\item We will cleanly seperate test modules from 
			application code.
		\item We will write tests that \emph{document} our
			code.
		\item We will write tests that run automatically.
		\item We will run our tests whenever we make a 
			change to be sure nothing was broken.
	\end{itemize}
\end{frame}

\begin{frame}
	\frametitle{Unit tests}

	\begin{itemize}
		\item Testing starts with \emph{unit tests}.
		\item Unit tests are written and run by developers.
		\item They test code at the lowest level of functionality.
		\item Basically, every function has a set of tests.
	\end{itemize}
\end{frame}

\begin{frame}
	\frametitle{Test Driven Development}

	Suppose you want to add a new feature to a class.
	If you are practicing TDD, you

	\begin{enumerate}
		\item Write tests that test the new feature.
			(They should fail.)
		\item Write code that passes the tests.
		\item \emph{Refactor}, or clean up your code.
	\end{enumerate}

\end{frame}

\begin{frame}
	\frametitle{Unit testing tools}
	\begin{itemize}
		\item There are heaps of libraries and 
			frameworks for writing unit testing modules.
		\item There is a family of testing tools that 
			are collectively called ``xUnit'' that are
			all based on Smalltalk's SUnit.
		\item This means that there is a common approach
			to unit testing that applies to many languages.
		\item Python's standard testing library, \texttt{unittest},
			is part of the xUnit family.  We will use it this week.
	\end{itemize}
\end{frame}

%\begin{frame}[fragile]
	%\frametitle{Example}
%
%\begin{verbatim}
%import random
%import unittest
%
%class TestSequenceFunctions(unittest.TestCase):
%
    %def setUp(self):
        %self.seq = range(10)
%
    %def test_shuffle(self):
        %# make sure the shuffled sequence does not lose any elements
        %random.shuffle(self.seq)
        %self.seq.sort()
        %self.assertEqual(self.seq, range(10))
%
        %# should raise an exception for an immutable sequence
        %self.assertRaises(TypeError, random.shuffle, (1,2,3))
%
    %def test_choice(self):
        %element = random.choice(self.seq)
        %self.assertTrue(element in self.seq)
%
    %def test_sample(self):
        %with self.assertRaises(ValueError):
        %random.sample(self.seq, 20)
        %for element in random.sample(self.seq, 5):
        %self.assertTrue(element in self.seq)
%
%if __name__ == '__main__':
    %unittest.main()
    %\end{verbatim}
    %\end{frame}


\end{document}
