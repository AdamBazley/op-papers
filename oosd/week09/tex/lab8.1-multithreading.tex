% Beamer slide template prepared by Tom Clark <tom.clark@op.ac.nz>
% Otago Polytechnic
% Dec 2012

\documentclass[10pt]{beamer}
\usetheme{Dunedin}
\usepackage{graphicx}
\usepackage{fancyvrb}
\newcommand\codeHighlight[1]{\textcolor[rgb]{1,0,0}{\textbf{#1}}}

\title{Multithreading}
\author[IN710]{OOSD}
\institute[Otago Polytechnic]{
  School of Information Technology \\
  Otago Polytechnic \\
  Dunedin, New Zealand \\
}
\date{}
\begin{document}

%----------- titlepage ----------------------------------------------%
\begin{frame}[plain]
  \titlepage
\end{frame}

%----------- slide --------------------------------------------------%
\begin{frame}[fragile]
  \frametitle{Enter and run this program}
\begin{Verbatim}[commandchars=\\\[\]]
import threading
import time

 def timer(name, delay, repeat):
     print("Timer " + name + " starting.")
     for i in xrange(repeat):
         time.sleep(delay)
         print(name + " " + time.ctime(time.time()))
     print("Timer " + name + " finishing.")

 if __name__ == "__main__":
     t1 = threading.Thread(target=timer, args=("Timer 1", 2, 5))
     t2 = threading.Thread(target=timer, args=("Timer 2", 4, 5))
     t1.start()
     t2.start()
 
     print("Main complete")

\end{Verbatim}

\end{frame}

\begin{frame}
	\frametitle{Threads}

	\begin{itemize}
		\item Threads are the smallest unit that can be
			scheduled by the operating system.
		\item Every process has at least one thread.
		\item Threads can share memory
	\end{itemize}
\end{frame}
%----------- slide --------------------------------------------------%
\begin{frame}[fragile]
	\frametitle{Syncronisation with \texttt{join}}
\begin{Verbatim}[commandchars=\\\[\]]
   t1 = threading.Thread(target=timer, args=("Timer 1", 2, 5))
   t2 = threading.Thread(target=timer, args=("Timer 2", 4, 5))
   t1.start()
   t2.start()
   print("Do stuff while threads run")
   t1.join()
   t2.join()
   print("All threads complete")

\end{Verbatim}

\end{frame}

\begin{frame}
	\frametitle{Uses for threads}

	\begin{itemize}
		\item \emph{Parallel processing}: We can break a 
			task into parts that can be performed in 
			parallel.  This can speed up the process.
		\item emph{Asynchonous processing}: We carry out a
			time consuming task in a seperate thread while
			the main process remains responsive.
	\end{itemize}
\end{frame}

\begin{frame}
	\frametitle{Thread class}

	In the following examples we will use the \texttt{Thread}
	class from the \texttt{threading} module. We do this to 
	create objects that run in their own threads.
\end{frame}

%----------- slide --------------------------------------------------%
\begin{frame}[fragile]
	\frametitle{Parallelisation example}
\begin{Verbatim}[commandchars=\\\[\]]
import os, re, threading

class PingCheck(threading.Thread):
    received_packages = re.compile(r"(\d) received")

    def __init__ (self,ip):
        threading.Thread.__init__(self)
        self.ip = ip
        self._successful_pings = -1

\end{Verbatim}

\end{frame}
%----------- slide --------------------------------------------------%
\begin{frame}[fragile]
	\frametitle{Parallelisation example}
\begin{Verbatim}[commandchars=\\\%\%]
def run(self):
    ping_out = os.popen("ping -q -c2 "+self.ip,"r")
    while True:
        line = ping_out.readline()
        if not line: break
        n_received = re.findall(self.received_packages,line)
        if n_received:
            self._successful_pings = int(n_received[0])


\end{Verbatim}

\end{frame}
%----------- slide --------------------------------------------------%
\begin{frame}[fragile]
	\frametitle{Parallelisation example}
\begin{Verbatim}[commandchars=\\\[\]]
def status(self):
    if self._successful_pings == 0:
        return "no response"
    elif self._successful_pings == 1:
        return "alive, but 50 % package loss"
    elif self._successful_pings == 2:
        return "alive"
    else:
        return "shouldn't occur"


\end{Verbatim}

\end{frame}

%----------- slide --------------------------------------------------%
\begin{frame}[fragile]
	\frametitle{Parallelisation example}
\begin{Verbatim}[commandchars=\\\%\%]
if __name__ == "__main__":
    check_results = []
    for suffix in range(1,255):
        ip = "10.25.1."+str(suffix)
        current = PingCheck(ip)
        check_results.append(current)
        current.start()

    for el in check_results:
       el.join()
       print "Status from ", el.ip,"is",el.status()

\end{Verbatim}

\end{frame}
%----------- slide --------------------------------------------------%
\begin{frame}[fragile]
	\frametitle{Asynchronous processing example}
\begin{Verbatim}[commandchars=\\\[\]]
class FileSaver(threading.Thread):
    def __init__(self, text, filename):
        threading.Thread.__init__(self)
        self.text = text
        self.filename = filename

    def run(self):
        f = open(self.filename, 'w')
        f.write(self.text)
        f.close()
        time.sleep(10)

\end{Verbatim}

\end{frame}
%----------- slide --------------------------------------------------%
\begin{frame}[fragile]
	\frametitle{Asynchronous processing example}
\begin{Verbatim}[commandchars=\\\%\%]
if __name__ == "__main__":
    files = []
    for i in xrange(3):
        msg = raw_input("Enter some text: ")
        fname = "file" + str(i) + ".txt"
        save = FileSaver(msg, fname)
        save.start()
        files.append(save)

    for f in files:
        f.join()
        print(f.filename + " was saved.")

\end{Verbatim}

\end{frame}


\end{document}
