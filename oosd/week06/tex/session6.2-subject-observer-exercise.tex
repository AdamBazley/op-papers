\documentclass{article}
\usepackage{graphicx}
\usepackage{wrapfig}
\usepackage{inconsolata}
\usepackage{enumerate}
\usepackage{hyperref}
\usepackage[margin = 2.25cm]{geometry}



\begin{document}


\title{Subject Observer Pattern Exercise\\IN710 Object Oriented System Development}
\date{}
\maketitle

\section*{Introduction}
In this exercise you will build an application that models the follower relationship
used by services like Twitter.

\section{The problem}
Your application will allow an individual to enter short messages.  It needs to
model two other individuals who ``follw'' the first one.  Whenever the first one
enters a new message, the message feeds of the other two must update to include 
newly entered message.  This kind of relationship can be modelled with a 
\emph{subject-observer} pattern.

\section{The task}
Write a PyQT application that does the following:

\begin{itemize}
	\item It will provide a text entry box in which a user
		can enter a short message.
	\item It will include two text area boxes that display
		the message feeds of two other people.
	\item When a new message is entered, that message
		will be prepended to the message feeds
		in the text area boxes.
	\item Each message is to be prefaced with a sensibly
		formatted timestamp.
	\item These behaviours are to be implemented with a 
		\texttt{User} class that supports a 
		follower/followed relationship.
	\item Include an appropriate unit test suite
		for your \texttt{User} class.
\end{itemize}

Commit your code and push it to your GitHub account.  We will
discuss your solutions in class next week.

\end{document}
