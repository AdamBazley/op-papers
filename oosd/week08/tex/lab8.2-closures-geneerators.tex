% Beamer slide template prepared by Tom Clark <tom.clark@op.ac.nz>
% Otago Polytechnic
% Dec 2012

\documentclass[10pt]{beamer}
\usetheme{Dunedin}
\usepackage{graphicx}
\usepackage{fancyvrb}
\newcommand\codeHighlight[1]{\textcolor[rgb]{1,0,0}{\textbf{#1}}}

\title{Closures and Generators}
\author[IN710]{OOSD}
\institute[Otago Polytechnic]{
  School of Information Technology \\
  Otago Polytechnic \\
  Dunedin, New Zealand \\
}
\date{}
\begin{document}

%----------- titlepage ----------------------------------------------%
\begin{frame}[plain]
  \titlepage
\end{frame}

%----------- slide --------------------------------------------------%
\begin{frame}[fragile]
  \frametitle{This works}

 
\begin{Verbatim}[commandchars=\\\[\]]
    x = 4
    def f():
        return x

    f()   # returns 4
\end{Verbatim}

\end{frame}
%----------- slide --------------------------------------------------%
\begin{frame}
  \frametitle{Closures}

  ``closures (also lexical closures or function closures) are a 
  technique for implementing lexically scoped name binding in 
  languages with first-class functions.'' (Wikipedia)

  Basically, they are a way of taking the previous example
  and exploiting first class functions to do something useful.

\end{frame}
%----------- slide --------------------------------------------------%
\begin{frame}[fragile]
  \frametitle{Constant function generator}

 
\begin{Verbatim}[commandchars=\\\[\]]
    def constant_builder(cval):
        def f():
            return cval
        return f

   four = constant_builder(4) 
       
\end{Verbatim}

Kind of interesting, but not that useful...
\end{frame}
%----------- slide --------------------------------------------------%
\begin{frame}[fragile]
  \frametitle{A counter}

  
\begin{Verbatim}[commandchars=\\\%\%]
    def counter_builder(start):
        count = [start]
        def f():
            val = count[0]
            count[0] += 1
            return val
        return f

   counter = counter_builder(1)
       
\end{Verbatim}

\end{frame}
%----------- slide --------------------------------------------------%
\begin{frame}[fragile]
  \frametitle{Memoisation}

  
\begin{Verbatim}[commandchars=\\\%\%]
    def factorial(n):
        if n == 0:
            return 1
        else
            return n * factorial(n-1)
       
\end{Verbatim}
Suppose we use this to compute 50!, and then we later 
compute 51!.  Wouldn't it be nice if we could have
saved the earlier result?
\end{frame}
%----------- slide --------------------------------------------------%
\begin{frame}[fragile]
  \frametitle{Memoisation}

  
\begin{Verbatim}[commandchars=\\\%\%]
    def fact_builder():
        memo = [1, 1]
        def f(in):
            try:
                return memo[n]
            except IndexError:
                result = n * f(n-1)
                memo[n] = result
                return result
        return f

   factorial = fact_builder()
       
\end{Verbatim}

\end{frame}

\begin{frame}
	\frametitle{Generator functions}

	Often we need an arbitrarily long sequence of values, like the counter function 
	we saw earlier.  We saw that we can produce these with closures, but it's a
	common enough situation so that Python provides \emph{generator functions}
	to handle this.
\end{frame}
%----------- slide --------------------------------------------------%
\begin{frame}[fragile]
  \frametitle{Generator example}

  
\begin{Verbatim}[commandchars=\\\%\%]
    def counter(n):
        count = 0
        while count < n:
            yield count
            count += 1

\end{Verbatim}
\end{frame}

\begin{frame}[fragile]
	\frametitle{Exercises}
	\begin{enumerate}
		\item Write a function that computes the nth \emph{Fibonacci number}.
		\item Use a closure to write a memoised version of your Fibonacci function.
			Use the \texttt{timeit} module to compare the speeds of your functions.
		\item Take the function below and reimplement it as a generator function.
	\end{enumerate}
\begin{Verbatim}[commandchars=\\\%\%]
from math import sqrt
def primes(n):
    if n == 0:
        return []
    elif n == 1:
        return [1]
    else:
        p = primes(int(sqrt(n)))
        no_p = {j for i in p for j in range(i*2, n, i)}
        p = {x for x in range(2, n) if x not in no_p}
        return p
\end{Verbatim}
\end{frame}
\end{document}
