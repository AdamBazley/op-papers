\documentclass{article}   	% use "amsart" instead of "article" for AMSLaTeX format
\usepackage[margin=0.5in]{geometry}                		% See geometry.pdf to learn the layout options. There are lots.
\geometry{a4paper}                   		% ... or a4paper or a5paper or ...

\usepackage[parfill]{parskip}    		% Activate to begin paragraphs with an empty line rather than an indent
\usepackage{graphicx}				% Use pdf, png, jpg, or eps with pdflatex; use eps in DVI mode
\usepackage{enumerate}								% TeX will automatically convert eps --> pdf in pdflatex		


\title{Lab 9.2:  Nagios Notifications\\ IN719 Systems Administration}
\date{}							% Activate to display a given date or no date

\begin{document}
\maketitle

\section*{Introduction}
Right now our Nagios servers are capable of sending us notifications via email when something happens.  This is good, but if something happens while we're not checking email, then we won't find out until later.  We would like to be able to push notifications to mobile devices so that we get alerts wherever we happen to be.  Probably the best approach is to set up a server that is capable of directly sending SMS messages, but in situations like ours that is not feasible.  Another approach is to use an SMS gateway service that accepts messages over the network via SMPP and forwards them via SMS.  The downside of this approach is that if your network goes down you also lose your ability to send a notification that the network is down, but sometimes that's a compromise we have to make.

If we're sending notifications over the network anyway, then there are some free messaging applications that we can use to send notifications to mobile devices.  One such application is \emph{WhatsApp}.  It is available for most mobile platforms and it has a pretty accessible API.  In this lab we'll set up our Nagios servers to send notifications using WhatsApp.

\section{Install WhatsApp}
You will need to install the WhatsApp mobile app on your device.  It's free for the first year of use and is available for most devices.  Install it through the appropriate channel for your device. See http://www.whatsapp.com for more information.  Register your account on your device.

\section{Install WhatsApp Nagios Plugin}
There is a Debian package file for a WhatsApp Nagios plugin available in the Week09 directory on the I: drive.  Install the package on your \texttt{mgmt} server.  Because of its dependencies, you should follow the following procedure:

\begin{enumerate}
  \item Copy the deb package to your server.
  \item Attempt to install it with the command \texttt{dpkg -i whatsapp-nagios-plugin-1.0.deb}.  You will probably get an error about unmet dependencies.
  \item Install the dependencies with the command \texttt{apt-get -fy install}.
  \item Now install the deb with the command \texttt{dpkg -i whatsapp-nagios-plugin-1.0.deb.}
\end{enumerate}

\section{Modify Contact Configuration}
With the plugin in place, you'll need to modify your contact configuration to send notifications via WhatsApp.  First, add a \texttt{pager} attribute to your contact.  Its value should be your mobile phone number, including the country code (e.g., 64211234567). Next, set your notifications commands to \texttt{notify-service-by-whatsapp} or \texttt{notify-host-by-whatsapp}.  Note that if you want to receive notifcations by both email and WhatsApp (a good idea), then you can use multiple notification commands seperated by commas.


\section{Disclaimers}

\begin{enumerate}
 \item I tested the Nagios plugin reasonably well, but it's entirely possible that there are bugs I didn't catch.
 \item The WhatsApp service is not completely reliable.  This is one reason why you should also get notifications by email, and you should check the web interface regularly.
\end{enumerate}
\end{document}
