% Beamer slide template prepared by Tom Clark <tom.clark@op.ac.nz>
% Otago Polytechnic
% Dec 2012

\documentclass[10pt]{beamer}
\usetheme{CambridgeUS}
\usepackage{graphicx}
\usepackage{fancyvrb}

\newcommand\codeHighlight[1]{\textcolor[rgb]{1,0,0}{\textbf{#1}}}

\title{Puppet on Windows}
\author[IN719]{Systems Administration}
\institute[Otago Polytechnic]{
  Otago Polytechnic \\
  Dunedin, New Zealand \\
}
\date{}

\begin{document}

%----------- titlepage ----------------------------------------------%
\begin{frame}[plain]
  \titlepage
\end{frame}

%----------- slide --------------------------------------------------%
\begin{frame}
  \frametitle{You can use Puppet to manage Windows}
  
\begin{itemize}
\item The puppetmaster must be *nix.
\item  The puppet agents can be Windows 7, Windows 8,  Server 2003 - Server 2012.
\item  The puppet agent version can be lower than the master (e.g., a 2.7 agent can talk to a 3.0 master), but not the reverse.
\item Some Puppet resources can be specified in a cross platform way, but
         \begin{itemize}
\item  some need to be specified in a platform specific way,
\item  some are platform specific.
\end{itemize}
\end{itemize}
\end{frame}

%----------- slide --------------------------------------------------%
\begin{frame}
  \frametitle{Installing the Windows agent}
  
\begin{itemize}
\item Download the msi from http://downloads.puppetlabs.com/windows or \\ the I: drive.
\item The msi's include all of the dependencies.
\item Edit the hosts file before you install.  The Agent needs to be able to resolve the hostname of the puppetmaster.
\end{itemize}


\end{frame}

%----------- slide --------------------------------------------------%
\begin{frame}
 \frametitle{Writing Manifests for Windows}
  
Puppet can manage the following resource types on Windows
\begin{itemize}
\item file
\item user
\item group
\item package
\item service
\item exec
\item host
\item scheduled\_task
\end{itemize}


\end{frame}

%----------- slide --------------------------------------------------%
\begin{frame}
  \frametitle{Windows File Paths}
  
Forward slash or black slash?

Use forward slashes for 
\begin{itemize}
\item file resource titles and paths
\item package resource paths
\item the command attribute of an exec resource (with exceptions)
\end{itemize}

You \emph{must} use back slashes for 
\begin{itemize}
\item the command of a scheduled\_task
\item the install\_options of a package resource
\end{itemize}


\end{frame}

%----------- slide --------------------------------------------------%
\begin{frame}
  \frametitle{Installing packages}
  
Puppet can install MSI and executable packages, but the file must be locally available on the agent.  You can either
\begin{itemize}
\item Manage the installer as a file resource, or
\item Place the installer on a mapped network drive, or
\item Specify a UNC path.
\end{itemize}
\end{frame}
%----------- slide --------------------------------------------------%
\begin{frame}[fragile]
  \frametitle{Installing packages}
  Examples

  \begin{verbatim}
    package { 'mysql':
        ensure          => '5.5.16',
        source          => 'N:/packages/mysql-5.5.16-winx64.msi',
        install_options => ['INSTALLDIR=C:\mysql-5.5'],
    }
    package { "Git version 1.8.4-preview20130916":
        ensure   => installed,
        source   => 'C:\\code\\puppetlabs\\temp\\windowsexample\\Git-1.8.4-preview20130916.exe',
        install_options => ['/VERYSILENT']
    }
  \end{verbatim}



\end{frame}


%----------- slide --------------------------------------------------%
\begin{frame}
  \frametitle{Users and Groups}
  Puppet can only directly manage local users and groups.  To manage domain users you will 
  need to use a Powershell script.

\end{frame}
%----------- slide --------------------------------------------------%
\begin{frame}
  \frametitle{Line Endings}

  Remember that Unix/Linux and Windows systems denote line endings differently.  If you 
  create a file resource on a *nix systems that will be used on Windows hosts, you may need 
  to modify the line endings.

\end{frame}



\end{document}
