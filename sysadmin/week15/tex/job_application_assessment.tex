\documentclass{article}   	% use "amsart" instead of "article" for AMSLaTeX format
\usepackage{geometry}                		% See geometry.pdf to learn the layout options. There are lots.
\geometry{letterpaper}                   		% ... or a4paper or a5paper or ...

\usepackage[parfill]{parskip}    		% Activate to begin paragraphs with an empty line rather than an indent
\usepackage{graphicx}				% Use pdf, png, jpg, or eps with pdflatex; use eps in DVI mode
\usepackage{enumerate}								% TeX will automatically convert eps --> pdf in pdflatex		


\title{IN719\\Systems Administration\\Job Application Assessment}
\date{}
\begin{document}
\maketitle
The final assessment for this paper will be a job application process.  The process will be modelled closely on real world hiring practices in the field and will have three parts.


\begin{description}
\item[Cover Letter and CV (40 marks)] Deliver these to Tom Clark's pigeonhole on level 3 of D Block.  These must be received by the deadline on the job posting.  Late applications will not be considered and you will receive a zero on this assessment.  You should prepare these documents as if you were applying for a real job and show the same attention to detail.  If you don't already have a CV, it's time to start maintaining one.  You can send drafts of these documents to Tom for review and feedback at any time before the deadline.

\item[Written Quiz (10 marks)] You will take a short (~20 minute) quiz during the second classroom session of the penultimate week.  The quiz is \emph{easy}.  It's purpose is to screen job candidates who are clearly unqualified.  If you have been participating in this paper, then you should not have to study for this quiz.

\item[Interview (50 marks)] \emph{You must receive at least 50\% marks on both the cover letter/CV and quiz portions of the exercise in order to get an interview.} The interviews will be scheduled for 30 minute blocks during the last week of term.  They will cover both technical and soft skills topics.  Dress for the interview will be ``business casual".  

\end{description}

The assessment will be marked according to the following scheme:
\begin{itemize}
\item 50-59:  The applicant meets minimal standards for hiring.  
\item 60-69:  The applicant is qualified for the job, but shows weakness in some areas
\item 70-79:  The candidate has no major weaknesses. 
\item 80-89:  The applicant is very strong in most areas, and may exceed expectations to some degree. 
\item 90-100: The applicant shows technical skills and professionalism far exceeding expectations.  
\end{itemize}




\end{document}
