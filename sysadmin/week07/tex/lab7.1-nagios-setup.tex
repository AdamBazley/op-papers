\documentclass{article}         % use "amsart" instead of "article" for AMSLaTeX format
\usepackage[margin=0.5in]{geometry}                             % See geometry.pdf to learn the layout options. There are lots.
\geometry{a4paper}                              % ... or a4paper or a5paper or ...

\usepackage[parfill]{parskip}                   % Activate to begin paragraphs with an empty line rather than an indent
\usepackage{graphicx}                           % Use pdf, png, jpg, or eps with pdflatex; use eps in DVI mode
\usepackage{enumerate}                                                          % TeX will automatically convert eps --> pdf in pdflatex                
\usepackage{hyperref}

\title{Lab 7.1:  Nagios Setup\\ IN719 Systems Administration}
\date{}                                                 % Activate to display a given date or no date

\begin{document}
\maketitle

\section*{Introduction}
In this lab we will configure our Nagios system to monitor our MySQL
database server.  You should have already created a Puppet module
for your Nagios server and used it to install Nagios on your 
\texttt{mgmt} server.

All tasks in this lab will be performed on your \texttt{mgmt} server.

\section{Enable logins to your Nagios web console}
When you installed the \texttt{nagios3} package it included a web based 
monitoring console that you can view at

\texttt{http://<your-mgmt-address>/nagios3}.

The only problem is that you can't log in.  We need to set up a password file
for Nagios with the command

 \texttt{sudo htpasswd -c /etc/nagios3/htpasswd.users nagiosadmin}

You will be asked to supply a password.  You will use that with the 
user name \texttt{nagiosadmin} to log onto the web console.

Now log in.  Loook in particular at Hosts and Hostgroups.  Nagios is configured
to monitor the local machine (\texttt{mgmt}) and nothing else.

\section{Configure Nagios using Puppet}

We want to be able to monitor the db server.  We will modify and add resources to
our nagios server puppet module's \texttt{config} class. 

1.  Create a contact for yourself (and others in your team).

\begin{verbatim}

nagios_contact { 'tclark':
              target => '/etc/nagios3/conf.d/ppt_contacts.cfg',
              alias => 'Tom Clark',
              service_notification_period => '24x7',
              host_notification_period => '24x7',
              service_notification_options => 'w,u,c,r',
              host_notification_options => 'd,r',
              service_notification_commands => 'notify-service-by-email',
              host_notification_commands => 'notify-host-by-email',
              email => 'root@localhost',
}

  \end{verbatim}

We will change that email address in a later lab, but for now we want to use
\texttt{root@localhost}.

2. Contacts aren't used until they appear in a relevant contact group.

\begin{verbatim}

nagios_contactgroup { 'sysadmins':
               target => '/etc/nagios3/conf.d/ppt_contactgroups.cfg',
               alias => 'Systems Administrators',
               members => 'tclark', 
}

  \end{verbatim}



3. Now we will define a nagios host for our \texttt{db} server.
This will cause Nagios to perform checks to see that our server
is up.

\begin{verbatim}

nagios_host { 'db.sqrawler.com':
                 target => '/etc/nagios3/conf.d/ppt_hosts.cfg',
                 alias => 'db',
                 address => '10.25.1.18',
                 check_period => '24x7',
                 max_check_attempts => 3,
                 check_command => 'check-host-alive',
                 notification_interval => 30,
                 notification_period => '24x7',
                 notification_options => 'd,u,r',
                 contact_groups => 'sysadmins',
}
\end{verbatim}


You'll have to change the IP address to match the address of
your \texttt{db} server.

4. Now define your service.  This basically tells Nagios
how you want MySQL servers to be monitored.

\begin{verbatim}

nagios_service {'MySQL':
              service_description => 'MySQL DB',
              hostgroup_name => 'db-servers',
              target => '/etc/nagios3/conf.d/ppt_mysql_service.cfg',
              check_command => 'check_mysql',
              max_check_attempts => 3,
              retry_check_interval => 1,
              normal_check_interval => 5,
              check_period => '24x7',
              notification_interval => 30,
              notification_period => '24x7',
              notification_options => 'w,u,c',
              contact_groups => 'sysadmins',
}
\end{verbatim}

And finally, define a host group that you will use to 
identify your MySQL servers to Nagios.

\begin{verbatim}
nagios_hostgroup{'db-servers':
              target => '/etc/nagios3/conf.d/ppt_hostgroups.cfg',
              alias => 'Database Servers',
              members => 'db.sqrawler.com',
}

\end{verbatim}


\section{Other steps}

There are a few housekeeping details to get things working properly.

\begin{verbatim}
sudo chown root:puppet /etc/nagios3/conf.d
sudo chmod 775 /etc/nagios3/conf.d 
\end{verbatim}

Have puppet configure your server and verify that the right files are in 
\texttt{/etc/nagios3/conf.d}.  Sadly, the permissions will be wrong.  
Fix them with

\begin{verbatim}
sudo chmod 644 /etc/nagios3/conf.d/* 
\end{verbatim}

If you have done everything right, your \texttt{db} server will appear to be up in Nagios,
but the MySQL service will appear to be down. We'll sort that out later.

\end{document}

