\documentclass{article}
\usepackage{graphicx}
\usepackage{wrapfig}
\usepackage{inconsolata}
\usepackage{enumerate}
\usepackage[margin = 2.25cm]{geometry}




\begin{document}

\begin{figure}
\includegraphics[width=30mm]{oplogo.png}
\end{figure}

\title{Course Directive\\IN515 Introduction to Networks\\Semester Two, 2014}
\date{}
\maketitle

\section*{Description}
This course will introduce you to fundamental networking concepts and technologies. You will cover the basics of network theory and develop the skills needed to implement a simple network. This course follows the curriculum for CCNA1 and is the first of four courses covering the material for the CCNA qualification.

\section*{Course Information}
\begin{itemize}
  \item 15 Credits
  \item No prerequisites
\end{itemize}

\section*{Lecturers}
\begin{tabular}{lrlr}

  % after \\: \hline or \cline{col1-col2} \cline{col3-col4} ...
  Tom Clark &    & Darrell Love &  \\
     Office: & D311 & Office: & D316\\
     Phone: & 470 4356 & Phone: & 479 3302\\
     Email: & \texttt{tom.clark@op.ac.nz} & Email: & \texttt{darrell.love@op.ac.nz}\\
\end{tabular}

\section*{Course Dates}
\begin{tabular}{ll}
Term 1 (10 weeks) & 21 July - 26 September\\
Mid semester break & 27 September - 12 October\\
Term 2 (6 weeks) & 13 October - 21 November\\
\end{tabular}

\section*{Learning Outcomes}
On completion of this paper you will be able to:
\begin{enumerate}
  \item Use network protocol models to explain the layers of communications in data networks;
  \item Design, calculate, and apply subnet masks and addresses;
  \item Build a simple Ethernet network using routers and switches;
  \item Employ basic cabling and network designs to connect devices;
  \item Perform basic router and switch configuration and verification;
  \item Analyse the operations and features of the transport and network layer protocols and services.
\end{enumerate}

\section*{Resources}
\begin{itemize}
 \item Course readings, exercises, quizzes and exams are available on the course web site at http://www.netacad.com
 \item Lab exercise documents are available on the I: drive.  You may print these yourself as needed or purchase a printed copy from the Polytech.
 \item Hardware and software required for labs and assessment are provided in the D312 and D313 labs.
 \item \emph{Packet Tracer} a Cisco network simulation tool is used for some activities.  You may obtain a copy for your own computer either on the I: drive or from the Cisco Web site.
\end{itemize}

%\pagebreak

\section*{Course Content and Schedule}
This schedule is pretty firm, although as always it is subject to change based on the needs of the class.
\\

\renewcommand{\arraystretch}{1.5}
\begin{tabular}{|l|c|l|l|l|}
\hline
 Week & Week Start & \multicolumn{1}{c|}{Topics}                   & Chapter & Exam \\ \hline
 1    & 21 Jul     & Introduction, Exploring the Network           &  1      &  1 \\ \hline
 2    & 28 Jul     & Configuring a Network Operating System        &  2      &  2 \\ \hline
 3    &  4 Aug     & Network Protocols                             &  3      &  3 \\ \hline
 4    & 11 Aug     & Network Access                                &  4      &  4  \\ \hline
 5    & 18 Aug     & Ethernet                                      &  5      &  5 \\ \hline
 6    & 25 Aug     & Network Layer                                 &  6      &  6 \\ \hline
 7    &  1 Sep     & Transport Layer                               &  7      &  7 \\ \hline
 8    &  8 Sep     & IP Addressing                                 &  8      &  * \\ \hline
 9    & 15 Sep     & IP Addressing, Subnetting                     &  8/9    &  8 \\ \hline
 10   & 22 Sep     & Subnetting                                    &  9      &  9 \\ \hline
 H1   & 29 Sep     & Holiday                                       &         &    \\ \hline
 H1   &  6 Oct     & Holiday                                       &         &    \\ \hline
 11   & 13 Oct     & Application Layer                             &  10     &  10 \\ \hline
 12   & 20 Oct     & Networks                                      &  11     &  *  \\ \hline
 13   & 27 Oct     & Networks                                      &  11     &  11 \\ \hline
 14   &  3 Nov     & Revision, Practice Exams                      &         &     \\ \hline
 15   & 10 Nov     & Theory Exam, SBA                              &         &  \\ \hline
 16   & 17 Nov     & SBA                                           &         &  \\ \hline
\end{tabular}

\newpage 

\section*{Assessment}
You are expected to take the chapter tests during the second class of the scheduled week. We
will have final exams during the last two weeks of class time.  A student may resit or make two
chapter exams. All theory tests are closed book assessments - no notes or books are permitted.

A student who resits the Final Theory Exam or Skills Based Assessment and passes it will be given 
a grade of 70\% on it.

Assessments are weighted as follows: 
\\
\\
\begin{tabular}{|l|c|}
\hline
Assessment                             &  Weighting \\ \hline
Chapter Exams                          &  20\% \\ \hline
Final Theory Exam                      &  30\% \\ \hline
Configuration Skills Based Assessment  &  10\% \\ \hline
Subnetting Skills Based Assessment     &  10\% \\ \hline
Final Skills Based Assessment          &  30\% \\ \hline
\end{tabular}

\section*{Criteria for Passing}
Cisco has its own grading scale, which is different to the grading scale used in the OP BIT. Your grade will be converted to the OP grading scale as shown in the following table:
\\

\begin{tabular}{|c|c|c|}
\hline
If your Cisco Weighted Score is &	Then your Grade is	& And your BIT Score is \\ \hline
90.0   100	                    &   A+	                & 95.0 \\ \hline
87.5 – 89.9	                    &   A	                & 87.5 \\ \hline
85.0 – 87.4	                    &   A-	                & 82.5 \\ \hline
82.5 – 84.9	                    &   B+	                & 77.5 \\ \hline
80.0 – 82.4	                    &   B	                & 72.5 \\ \hline
77.5 – 79.9	                    &   B-	                & 67.6 \\ \hline
75.0 – 77.4	                    &   C+	                & 62.5 \\ \hline
72.5 – 74.9	                    &   C                   & 57.5 \\ \hline
70.0 – 72.4	                    &   C-	                & 52.5 \\ \hline
\end{tabular}
\vspace{3mm}

You must also obtain a passing mark (70\% or higher) on the Final Theory Exam in order to pass this paper.

\section*{Course Requirements and Expectations}
\subsection*{Attendance}
\begin{itemize}
 \item Students are expected to attend all classes, both lectures and labs.
 \item If you miss a class you should get notes from another student.
 \item If you cannot attend for two or more consecutive sessions, contact the lecturer.
 \item You must be present for assessments on the due date at the correct time.
\end{itemize}

\subsection*{Communication}
Important announcements and discussions about the course, assessments, and scheduling may take place during class sessions.  It is your responsibility to be informed about them.  If you cannot attend a class session, be sure to check with another student.

Your student email is an official communication channel. It is your responsibility to regularly check your student email for important course related material, including changes to class scheduling or assessment details. Not checking will not be accepted as an excuse.

You can manage your email at the Student Hub and download the instructions for forwarding your email at http://www.op.ac.nz/students/student-hub/

\subsection*{Polytechnic Closure}
In the event that the Polytechnic is closed or has a delayed opening because of snow or bad weather, you should not attempt to attend class if it is unsafe to do so. It is possible that your instructor will not be able to attend either, so classes will not physically be meeting. However, this does not become a holiday. Rather, material will be available on the Cisco Academy web site covering the material for classes affected by the closure. You are responsible for any material presented in this manner. Information about closure will be posted on the Otago Polytechnic facebook page https://www.facebook.com/OtagoPoly.

\subsection*{Group Work and Originality}
Students in the Bachelor of Information Technology degree are expected to hand in original work.  Students are encouraged to discuss
assignments with their fellow students.  However, all assignments are to be completed as individual works unless group work is explicitly involved.
Failure to submit your own unique work will be treated as plagiarism.

\subsection*{Referencing}
Appropriate referencing is required for all work.  Referencing standards will be specified by your instructor.

\subsection*{Plagiarism}
Plagiarism is submitting someone else's work as your own.  Plagiarism offences are taken seriously and an
assessment that has been plagiarised may be awarded a zero mark.  A definition of plagiarism is in the Student Handbook,
available online or at the school office.

\subsection*{Submission Requirements}
All assignments are to be submitted by the time, date, and method given when the assignment is issued.

\subsection*{Extensions}
Extensions are only available for unusual circumstances.  These must be applied for, and approved, prior to the submission deadline.

\subsection*{Impairment}
In case of sickness contact your lecturer or year co-ordinator as soon as possible, preferably before the test or
assignment is due.  The policy regarding the granting of a mark that considers impaired performance requires a medical
certificate and a medical practitioners signature on a form. You may should refer to the guide on impaired performance
on the student handbook.

\subsection*{Appeals}
If you are concerned about any aspect of your assessment, please approach the lecturer in the first instance.  We support
an open door policy and aim to resolve issues promptly.  Further support is available from the Programme
Manager and Head of School. Otago Polytechnic has a formal process for academic appeals if necessary.

\subsection*{Other Documents}
Regulatory documents relating this course can be found on the Polytechnic website.




\subsection*{Special Resources and Requirements}
If you have any special needs, whether they relate to the course material, the exercises, the assessment, or anything in the course -
then \textit{please} let your instructor know as soon as possible.

\end{document}
